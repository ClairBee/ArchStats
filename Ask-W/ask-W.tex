\documentclass[10pt,fleqn]{article}
\usepackage{/home/clair/Documents/mystyle}

%----------------------------------------------------------------------
% reformat section headers to be smaller \& left-aligned
\titleformat{\section}
	{\normalfont\bfseries}
	{\thesection}{1em}{}
	
\titleformat{\subsection}
	{\normalfont\bfseries}
	{\llap{\parbox{1cm}{\thesubsection}}}{0em}{}
%======================================================================

\begin{document}

\section*{Things to ask Wilfrid on August 24th}
\vspace{-10pt}

%\todo{\nb{CHECK WITH CHRIS S} Genlis site map: is the main block one large building, or several smaller buildings?}

\todo{Abstract: Should I use the one given when we chose the projects, or do I need to write one tailored to what I've actually done?}

%\todo{Which bibliography style is to be preferred? (actual style name would really help)}

\todo{Re standard-unit analysis / linear feature analysis: Feasible to include this as a brief write-up in 'extensions' or is this a waste of time? Could outline the general approach that would be suggested \& maybe give an example of the beginning of the work. But may not have time to get through the whole thing.}

\todo{Feels like quite a lot of regurgitated formulae at the moment - two distributions, a whole bunch of tests. Am worried this is a little too thorough.}

\todo{There are an awful lot of formulae taken from books. How should I cite this? Include a citation for everything I refer to, or just once at the beginning of the section? Especially for tests, do I need to give a citation for every test I introduce that is not of my own devising?}

\todo{When outlining procedure for identification of gridding, I've used simulated data set to illustrate what data 'should' look like. Would it be better to keep this type of analysis for the two case studies?}

\todo{In the R code, should I include absolutely everything, to the point where someone could just pick it up, run it and recreate everything (including creating example plots \& outputting to pdf) or should I only include the analysis, eg case study 1 \& 2, plus separate copies of functions used?}

\todo{Perturbation?!? - haven't done anything with perturbation of the data yet. If I can get a decent way to identify regions of the site with shared orientation, is it really necessary? Feels like it may be a bit redundant. \nb{Would W still recommend that I put in a perturbation-based sensitivity study or some such?} - I don't have anything to answer the "but what if this arose by chance" bridge.}

\todo{No method found to quantify the level of gridding. Re-read `in search of quanta' - how were the conclusions framed there?}

\todo{Final step: identifying evidence of global gridding. Should I be aiming for a single measure or can I go for something more descriptive? Idea: split site into grid (similar to Fisher PCA analysis), compare subset of points within each grid square for common mean direction, use this to cluster regions of the grid.}

\todo{Monte Carlo approach as in Hunting Quanta: probably not applicable here}





\end{document}