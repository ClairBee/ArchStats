\documentclass[10pt,fleqn]{article}
\usepackage{/home/clair/Documents/mystyle}

%----------------------------------------------------------------------
% reformat section headers to be smaller \& left-aligned
\titleformat{\section}
	{\normalfont\bfseries}
	{\thesection}{1em}{}
	
\titleformat{\subsection}
	{\normalfont\bfseries}
	{\llap{\parbox{1cm}{\thesubsection}}}{0em}{}
%======================================================================

\begin{document}

\section*{Things to ask Wilfrid on August 24th}
\vspace{-10pt}

\todo{\nb{CHECK WITH CHRIS S} Genlis site map: is the main block one large building, or several smaller buildings?}

\todo{Abstract: Should I use the one given when we chose the projects, or do I need to write one tailored to what I've actually done?}

\todo{Which bibliography style is to be preferred? (actual style name would really help)}


\todo{No method found to quantify the level of gridding. Re-read `in search of quanta' - how were the conclusions framed there?}

\todo{There are an awful lot of formulae taken from books. How should I cite this? Include a citation for everything I refer to, or just once at the beginning of the section? Especially for tests, do I need to give a citation for every test I introduce that is not of my own devising?}

\todo{In the R code, should I include absolutely everything, to the point where someone could just pick it up, run it and recreate everything (including creating example plots \& outputting to pdf) or should I only include the analysis?}

\todo{Haven't included every possible formula eg. tests of uniformity, I've introduced them, but haven't always given the test statistic and null distribution: was seeming a bit repetitive. Is this okay or should I put in every formula?}

\end{document}