\documentclass[10pt,fleqn]{article}
\usepackage{/home/clair/Documents/mystyle}

%----------------------------------------------------------------------
% reformat section headers to be smaller \& left-aligned
\titleformat{\section}
	{\normalfont\bfseries}
	{\thesection}{1em}{}
	
\titleformat{\subsection}
	{\normalfont\bfseries}
	{\llap{\parbox{1cm}{\thesubsection}}}{0em}{}
%======================================================================

\begin{document}

\section*{Things to ask Wilfrid on August 24th}
\vspace{-10pt}

\todo{Can I justify filtering the data using $\varepsilon$-blunt shapes/2nn angle?}

\todo{Can I justify filtering the data by removing remote points?}

\todo{No method found to quantify the level of gridding. Re-read `in search of quanta' - how were the conclusions framed there?}

\todo{Bias-corrected parameters etc: feels like I'm just parroting formulae (not a lot to say: they've been shown to be biased, so I've used the corrections). But for confidence intervals, I haven't quoted the formulae, just pointed the reader to where they can find them. Is this okay?}

\todo{There are an awful lot of formulae taken from books. How should I cite this? Include a citation for everything I refer to, or just once at the beginning of the section? Especially for tests, do I need to give a citation for every test I introduce that is not of my own devising?}

\todo{Which is the best way to proceed: introduce the theory and tests, then lay out a procedure separately? Or introduce theory and tests as they arise in the procedure? \nb{Think I probably should have gone with the former...}}

\todo{Should I be making direct reference to the named functions I've created, or just describing them and putting the code in the appendix?}

\todo{In the R code, should I include absolutely everything, to the point where someone could just pick it up, run it and recreate everything (including creating plots \& outputting to pdf) or should I only include the analysis?}

\todo{Haven't included every possible formula eg. tests of uniformity, I've introduced them, but haven't always given the test statistic and null distribution: was seeming a bit repetitive. Is this okay or should I put in every formula?}

\todo{Should I include actual numbers in eg. my `gridding' section? Still explaining the procedure, would it be better to keep actual numerical evaluation until we have some real data?}

\todo{I've included simulated data examples to illustrate how the data `should' look. Is this a good idea or should I just use the case studies for that?}

\todo{I don't think it's really necessary to test for common concentration or common distribution: If we have two non-uniform subsets with reflective symmetry and a common mean, the exact distribution is not important. Although that does rather waste all the time I've spent writing up the circular statistics.}

\end{document}