\documentclass[10pt,fleqn]{article}
\usepackage{/home/clair/Documents/mystyle}

%----------------------------------------------------------------------
% reformat section headers to be smaller \& left-aligned
\titleformat{\section}
	{\normalfont\bfseries}
	{\thesection}{1em}{}
	
\titleformat{\subsection}
	{\normalfont\bfseries}
	{\llap{\parbox{1cm}{\thesubsection}}}{0em}{}
%======================================================================

\begin{document}

\section*{Things to ask Wilfrid on August 24th}
\vspace{-10pt}

\todo{Can I justify filtering the data using $\varepsilon$-blunt shapes/2nn angle?}
\todo{Can I justify filtering the data by removing remote points?}

\todo{Bias-corrected parameters etc: feels like I'm just parroting formulae (not a lot to say: they've been shown to be biased, so I've used the corrections). But for confidence intervals, I haven't quoted the formulae, just pointed the reader to where they can find them. Is this okay?}

\todo{Should I be making direct reference to the named functions I've created, or just describing them and putting the code in the appendix?}

\todo{In the R code, should I include absolutely everything, to the point where someone could just pick it up, run it and recreate everything (including creating plots \& outputting to pdf) or should I only include the analysis?}

\todo{Haven't included every possible formula eg. tests of uniformity, I've introduced them, but haven't always given the test statistic and null distribution: was seeming a bit repetitive. Is this okay or should I put in every formula?}

\end{document}