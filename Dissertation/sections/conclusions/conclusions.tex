\documentclass[../../ArchStats.tex]{subfiles}

\begin{document}

\section{Summary \& Conclusions}

\nb{von Mises distribution is not generally the best fit because it is unable to deal with the combination of uniform `noise' points and concentrated spikes that we tend to see arising from this type of data. A Jones-Pewsey may fit the data better, but can generally be replaced with a mixture of von Mises distributions. While a $k$-von Mises mixture may be the absolute best fit of the candidate models to the data, a more useful model in terms of interpretation is probably one in which one component is fixed as a circular uniform distribution, describing the density of those points that can't be said to belong to any particular orientation.}

\todo{Give a summary of the procedure used: cleaning the data to extract post-holes, identifying orientation, identifying gridding (rather than a single dominant linear direction), identifying gridding across the whole site}

\todo{General remarks: more effective over simpler sites such as Genlis, where a fairly clear result can be obtained. In Catholme, may only be able to tell us the general orientation. However, even this may be useful to archaeologists? \nb{Have demonstrated a potential extension whereby multiple grids may be identified, but further work is required to refine this process.}}

\todo{In describing the Genlis site in particular, we have found a distribution to describe a v concentrated set of angles with a shared orientation. Want to compare this to other sites to verify how generally applicable this distribution is. Might be able to use it to describe an axial distribution in other sites where the data is not so concentrated: if we remove these points and replace with uniform, do we reveal a further axis?}
\end{document}