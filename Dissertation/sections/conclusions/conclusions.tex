\documentclass[../../ArchStats.tex]{subfiles}

\begin{document}

\section{Conclusions}
\label{sec:concl}

The proposed procedure complements the existing visual  approach to analysis of underlying grid patterns in the survey data, providing a largely autonomous procedure  to identify any evidence of a common grid, and to quantitatively assess the degree to which the data supports the hypothesis of either a global or local grid across the site.  

\subsection{Data cleaning}
Application of all stages of the process may not always be required to obtain a set of post-hole features, but since each step is subtractive, over-filtering can only result in an increased number of false negatives, minimising the risk of more potentially damaging (because potentially correlated) false positive identifications. Each step is based on attributes that are invariant under changes to the map's scale or to the attributes of other features in the site, in a way that filters based on the absolute size of the features would not be, and this leads to a process that can be applied generally to plans of any scale or resolution where the post-hole features are represented as solid points.


In the first case study, all of the cleaning steps were required, excluding 380 features and leaving a final feature set of 243 points after remote points were removed. Although we have no definitive objective classification of the features to enable us to assess which of the features were correctly classified as post-hole or not, a close visual comparison of the original site plan to the 243 selected points reveals  14 false positives (6 points on the site boundary; 4 smudges; 4 annotations, of which 3 were decimal points in labels). 8 false negatives were also identified. The number of mis-classified points is fairly small, with less than 6\% of the post-holes in the plan incorrectly identified and 3\% of the post-holes incorrectly excluded, and cannot be considered significant in terms of the effect on the effect on the angular analysis. Due to the complexity of the Catholme plan, a similar assessment is not feasible; however, we can say with certainty that all of the plan's annotations were removed by the cleaning process, and that the points included in the angular analysis therefore represent the mid-points of physical features of the site, whether post-holes or very small linear structures. 

Having tested the data cleaning process on two quite different sites, then, we have found it to be a robust approach requiring no site-specific parameter tuning, able to remove annotations even where they are of similar dimensions to the post-hole features that we wish to retain for analysis, and able to cope with JPEG images of different qualities and resolutions, including secondary scans of printed archaeological surveys. 

\subsection{Angular analysis}
Only two case studies have been undertaken thus far, so any statements about the effectiveness of this approach must be rather cautious. 

In both cases examined, the process produced a result that concurs with our initial predictions; for the Genlis site, clear evidence of an underlying grid was found, while in the Catholme site, only a single small region was found to show any evidence of a single grid orientation. These two cases are rather extreme examples, and before stronger statements can be made about the effectiveness of the approach, we would need to carry out a thorough investigation of the robustness and sensitivity of both tests. A question of particular importance here is that of the exact degree of angular separation required between the mean directions of two gridded regions before they are no longer deemed to share a common mean direction. This will allow us to obtain a more nuanced assessment of just how closely aligned the various regions of the site are, rather than relying on a significance test.

The successful identification of a global grid in the  Genlis post-holes produced a very concentrated set of angles with a shared orientation. At present, having only observed one site with a strong orientation, we have no way to assess how typical this distribution might be, and therefore how significant it might be. Further case studies are required for comparison, to allow us to develop a clearer idea of the level of concentration that should be considered meaningful.

%{While evidence of a common grid was detected, close inspection of Figure~\ref{fig:Genlis-ph-clusts} reveals that the winner-takes-all clustering based on the fitted uniform-von Mises mixture model has classified as noise some post-holes that, to the human eye at least, appear to lie directly in between two points in the same cluster, on a line very close to that of the cluster's orientation (consider for example, the pair of points at approximately (60, 40). This serves as a cautionary note: this approach was not intended as a tool to accurately identify features, but to identify evidence of a common grid orientation across the site, and this has been achieved. The method might be easily adapted to provide a means of algorithmically  identifying structures within the site, but this would require consideration of the spatial relationships between points and the similarity of their angles together; we have purposefully separated the two in order to compare the two types of clustering.}

Where multiple grids are present in a site, a mixture model fitted using the E-M algorithm will be able to identify their modal directions only if they are well separated; where multiple grids are superposed, or their axes are very similar, it is unlikely to successfully distinguish between them. A more sensitive approach is therefore likely to be required in those sites, like  Catholme, where different alignments coexist in a small area. A possible approach to this type of problem is proposed in section~\ref{sec:multiple-grids}.

%====================================================================================
%%\hrulefill


%\todo{Robustness of approach}

%\todo{Limitations and usefulness}

%\todo{Usefulness of approach}

%\todo{Good first step. Further refinements probably required.}

%\nb{von Mises distribution is not generally the best fit because it is unable to deal with the combination of uniform `noise' points and concentrated spikes that we tend to see arising from this type of data. A Jones-Pewsey may fit the data better, but can generally be replaced with a mixture of von Mises distributions. While a $k$-von Mises mixture may be the absolute best fit of the candidate models to the data, a more useful model in terms of interpretation is probably one in which one component is fixed as a circular uniform distribution, describing the density of those points that can't be said to belong to any particular orientation.}


\end{document}