\documentclass[../../ArchStats.tex]{subfiles}

\begin{document}
\section{Introduction}

\subsection{Project motivation}
The data and motivation for this study were provided by the Planning in the Early Medieval Landscape project \cite{PEML}. As part of their investigation into the extent of planning in the early Medieval period (AD 410-1066), the team are interested in evidence of planned settlements on  a large scale, and particularly in evidence that settlements may have been laid out over a common grid. It has been suggested that a grid-planning system may have been in use in a region known as the Central Province during the period, implying a more sophisticated level of surveying technology than is currently generally assumed to have been in use. Some preliminary work has been done in support of this hypothesis (\cite{Blair2013}), with much of the evidence derived from expert assessments of 19th-century maps of the settlements. We aim to develop an approach by which an objective assessment of the strength of the evidence for grid planning in certain key sites can be made.

\subsection{Questions of interest}

We will consider the evidence of the post-hole features contained within the site plans. These post-holes are all that remains of posts that were once set into the ground as support for walls and buildings, so indicate the edges of structures, and so post-holes that form part of the same wall will be aligned with one another. Our goal is to develop a methodology by which post-holes can identified from  a digitized image of an excavation plan and converted into a set of points in two-dimensional space, and those points assessed for evidence that they lie on an underlying grid pattern common to the whole site. 

Where the buildings conform to a grid, such as the simulated example shown in Figure~\ref{}, we can expect the orientations of their post-holes to follow a 4-cyclic distribution with pronounced peaks, similar to the one shown in Figure~\ref{}; `folding' the angles into the same quarter-circle by taking the orientations modulo $\pi/2$, we expect to see a symmetric unimodal distribution similar to Figure~\ref{}, and it is this type of distribution that will form the basis of our analysis. Differently-aligned structures in the same site will produce multi-modal distributions or - if the alignments are similar - a blurred (less-concentrated) or, if the structures are of different sizes, a skewed unimodal distribution\nb{figures?};  post-holes that are not closely aligned  to any particular structure will add generally uniform noise to the data set. All of these factors must be considered when attempting to detect evidence of gridding. Finally, the degree to which the grid can be detected across the whole site must be assessed.




\subsection{Project outline}
We will focus on the distribution of angular measurements taken between post-holes, for which techniques specific to circular data are required; since these distributions are generally not as familiar as their cousins on the real line, we begin by introducing the descriptive statistics required to identify and analyse patterns in the directional data. Two circular distributions are introduced - the von Mises or Circular Normal model, and the more flexible Jones-Pewsey model - followed by a mixture model, whose parameters may have a more direct interpretation.

In Section~\ref{sec:data-cleaning}, various techniques for extraction of features of interest from the digitized site map are proposed, by which groups of pixels might be filtered according to their morphological characteristics to obtain a set of points representing the locations of post-holes in the site. We then describe a method by which the orientation of each of the points will be obtained in circular form.

Section~\ref{sec:gridding} begins with the modelling process, then introduces tests to compare the similarity of multiple angular samples, which will be needed both to confirm the grid's perpendicularity and to confirm the presence of a common grid in multiple regions of the site. Finally we describe how separate regions of the grid will be identified for comparison.

The techniques and models described are applied in two case studies, one a relatively small site with visibly aligned post-holes, and one a much larger site with no immediately discernible pattern. The results of applying the method to these two very different sites are discussed briefly in section~\ref{}, and a number of possible extensions to the project are proposed.


\textbf{A note on coding}

All of the data analysis was carried out in R. Many of the required functions can be found in the \texttt{circular} package; additional functions, particularly concerning the Jones-Pewsey distribution and comparison of multiple samples, were created based on the examples given in \cite{Pewsey2014} and can be downloaded and installed directly from Github. The URL is given in Appendix~\ref{app:R-code}, along with a copy of the code.
\end{document}