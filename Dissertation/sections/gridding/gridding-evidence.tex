\documentclass[../../ArchStats.tex]{subfiles}

\begin{document}

\section{Finding evidence of gridding}

\nb{Run a simulation of the kind of data we expect/hope to find. Plot a few perturbed buildings and get the angles - use this as an example of what we hope to find \& what the distribution should look like when we do}

\subsection{Extracting angles from post-holes}
`Doubling' the angles to allow a unimodal distribution to be fitted.

\subsection{A note on axial data}
Under our null assumption that the measured angles will tend to be concentrated around the four axes of an underlying grid, we will consider the angles $\phi, \phi + \pi/2, \phi + \pi,$ and $\phi + 3\pi/2$ to be part of the same axis, and so we wish to analyse them as the same angle $\theta = \phi \text{ modulo }\pi/2$.
To this end we will follow Fisher's approach to $p$-axial data \cite{Fisher1993} using $p=4$: the raw angles $\phi_i$ are transformed to $4\theta_i \text{ modulo } 2\pi$ - equivalent to taking $4 \times (\phi_i \text{ modulo }\pi/2)$ giving a unimodal data set to which we can fit a circular distribution. Raw angles that share a perpendicular orientation - that is, angles that are directly opposed or perpendicular to one another - are thus mapped to the same angle, allowing clearer analysis of the direction of the underlying grid.

Once a distribution has been fitted, the mean sample direction obtained will be back-transformed by dividing by 4, to give the direction of one (and hence, trivially, all) of the axes of the grid; to allow for easier comparison between the back-transformed angles, these will be given in degrees, rather than radians. Measures of dispersion such as the mean resultant length $\bar{R}$ will not be back-transformed, as per Fisher's recommendation, but will be given in terms of the transformed data.

\subsection{Tests for symmetric unimodality}
Demonstrate that it's appropriate to even fit one of these distributions

What do we do if we don't find symmetry/unimodality in the data?

\subsection{Identifying the dominant direction}


\subsection{Testing for similarity of multiple distributions}

\subsubsection{Linearity vs perpendicularity}

\subsubsection{Localized vs global gridding}

\end{document}