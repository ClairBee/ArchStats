\documentclass[../../ArchStats.tex]{subfiles}

\graphicspath{{/home/clair/Documents/ArchStats/Dissertation/sections/CS2-Catholme/img/}}

\begin{document}
\section{Case Study 2: Catholme}
\label{sec:CS2}

Our second case study is a plan of the dig site at Catholme. A more recent excavation than Genlis, the JPEG was generated directly from the CAD data rather than a scanned printed image; as a result, the image itself is much cleaner. However, the site is much larger and more complex than the one at Genlis, with post-hole structures and linear structures mixed together. In addition, the PEML team suspects that there may have been several phases of construction at this site, suggesting that there may not be a single common orientation. \nb{expand this a little}

A copy of the R code used to generate the plots and analysis can be found in Appendix \ref{app:CS2}.

 \begin{figure}[!ht]
 \begin{minipage}[t]{0.47\textwidth}
 \caption{JPEG plan of Catholme site. The plan has minimal annotations and, having been taken directly from CAD data, is a strictly black-and-white image.}
 \centering
 \includegraphics[scale=0.2]{Catholme-cropped.jpg}
 \end{minipage}
 \hfill
	\begin{minipage}[t]{0.47\textwidth}
	\caption{Final post-hole set extracted for Catholme site. Only dense features were retained; remote points have been removed from the set.}
	\label{fig:Catholme-f-ext-postholes}
	\centering
	\includegraphics[scale=0.35]{Catholme-postholes.pdf}
    \end{minipage}
 \end{figure}
 
\subsection{Point extraction}
Identifying potential post-hole features is rather more straightforward here then in the first case study. We begin in the same way, by dividing dense features - in which more than 55\% of the shape's bounding square is coloured - from less dense ones, and excluding the latter from our set of post-hole candidates. In fact, this single step is sufficient to extract a useable set of post-hole features in this case; the few annotations that appear are in a finer font than those on the Genlis scan, and are immediately excluded, with the exception of the two `e's in the short perch scale label - and these two features are excluded when remote features are removed in the next step. For confirmation, features were excluded according to whether their shape was changed by a morphological closing - removing 28 features out of 979 remaining - and the function to fill dotted boundaries was run, removing a further 4 points. This trial serves as confirmation that the additional steps were neither necessary nor damaging in this case, removing only a handful of the available points. The final set of 915 post-holes identified is shown in Figure~\ref{fig:Catholme-f-ext-postholes}, with the 64 points excluded by distance filtering highlighted in red.




\subsection{Fitting an angular distribution}

The angles extracted from the final post-hole set are shown in Figure~\ref{fig:Catholme-angles}. The raw angles $\boldsymbol{\phi}$ in Figure~\ref{fig:Catholme-angles-raw} again show a 4-cyclic pattern, with the kernel density (smoothed using the same bandwidth, 30, as the Genlis data) having a rounded quadrilateral shape, suggesting that the post-holes do follow a perpendicular grid, rather than a linear one. However, the transformed angles in Figures~\ref{fig:Catholme-angles-trans-circ} and~\ref{fig:Catholme-angles-trans-linear} are less concentrated, having a smaller peak and a much higher proportion of the density lying far from the modal direction.

\begin{figure}[h!]
\centering
\caption{Histograms of raw angles $\boldsymbol{\phi}$ and transformed angles $\boldsymbol{\theta}$, with kernel density estimate and, where appropriate, densities of candidate models overlaid for reference. The  legend is common to both representations of $\boldsymbol{\theta}$.}
\label{fig:Catholme-angles}
%
\begin{subfigure}[t]{0.3\textwidth}
\caption{Raw angles $\boldsymbol{\phi}$\\ \textcolor{white}{spacer}}
\label{fig:Catholme-angles-raw}
\includegraphics[scale=0.35]{Q-circ-plot.pdf}
\end{subfigure}
%
\begin{subfigure}[t]{0.3\textwidth}
\centering
\caption{Transformed angles $\boldsymbol{\theta}$\\ \textcolor{white}{spacer}}
\label{fig:Catholme-angles-trans-circ}
\includegraphics[scale=0.35]{Q4-circ-plot.pdf}
\end{subfigure}
%
\begin{subfigure}[t]{0.3\textwidth}
\caption{Linear histogram of $\boldsymbol{\theta}$ \\(centred at $\bar{\theta}$)}
\label{fig:Catholme-angles-trans-linear}
\includegraphics[scale=0.3]{Q4-linear-plot.pdf}
\end{subfigure}
\end{figure}

Tests of uniformity of $\boldsymbol{\theta}$ confirm that the data is unimodal, with the Rayleigh test producing a result of $p = 0$, and Kuiper's and Watson's tests $p < 0.1$.  However, Pewsey's test of reflective symmetry gives $p = 0.019$, rejecting the null hypothesis that the data has reflective symmetry about some unspecified direction at the 2\% significance level, and suggesting that a symmetric distribution may not be appropriate here.


\subsubsection{Parameter estimation and model selection}

The bias-corrected estimate of $\bar{\beta}_2$ confirms the result of the test of reflective symmetry: the 95\% confidence interval does not contain 0. However, the estimated value is -0.064, with the interval's upper limit just -0.0047; furthermore, the circular median lies at 5.395, well within the confidence intervals for all of the estimates of $\mu$, suggesting that the degree of skew is very small; we may still be tempted to attempt to fit a unimodal candidate distribution regardless.  Bootstrap goodness-of-fit tests using the circular probability integral transform method (Section~\ref{sec:unif-tests}) give a final confirmation that this approach would not be appropriate: the von Mises maximum-likelihood candidate is rejected at the 2\% significance level with $p = 0.010$ (Kuiper) and $0.016$ (Watson), while the Jones-Pewsey model is rejected with at the 5\% level with $p = 0.018$ and $p =  0.045$.

There exist families of circular distributions whose densities are asymmetric, and in particularly whose densities have only a low level of asymmetry\cite[4.3.11]{Pewsey2014}, so would be likely fit this data set well. However, it is hard to justify fitting such a distribution here: our aim is not simply to fit a distribution of some kind to the data, but to identify whether the angles exhibit evidence of a single unimodal distribution, which would indicate a large subset of post-holes with a shared orientation across the site, and thus reflect some evidence of a common grid. We have found no evidence of any such universal grid at the Catholme site. 

However, one possible explanation for both the skewness and the low concentration of the data may be the presence of more than one underlying grid orientation in the post-holes, either superposed in the same area or occupying different regions of the site. We will therefore proceed using a similar approach to that used to identify similarly-gridded regions within the Genlis site (Section~\ref{sec:Gen-local-grids}): dividing the post-holes into subsets of densely-clustered points, and treating each subset as if it were a new site.


\subsection{Evidence of local gridding}
The DBscan algorithm, with MinPts = 4 and $\varepsilon = 5$ as before, identifies 36 distinct clusters and 89 points not belonging to any particular cluster (for ease of reference, since none of the clusters were found to contain the same number of points, the clusters will be labelled with their sizes rather than the arbitrary numbering system obtained when generating the clusters). 8 of those clusters contain 25 or more points: model fitting in smaller samples than this tends to be rather unreliable - particularly when the data is sub-divided to test for perpendicularity - so only the larger clusters will be considered further. These 8 clusters contain 572 post-holes: a little under two-thirds of the 915 points on the plan. In one cluster (25), uniformity was not rejected at the 5\% level by either the Rayleigh or the Watson test; in two (44 and 39), reflective symmetry was rejected, leaving five clusters of points, containing just 464 (51\%) of the available post-holes (Figure~\ref{fig:Cat-clusters}). The raw angles $\boldsymbol{\phi}_C$ in each cluster in Figure~\ref{fig:clust-circ} show little evidence of the tell-tale quadrilateral shape that indicates a strong perpendicular axial orientation; dividing each cluster's angles into quadrants according to the method outlined in Section~\ref{sec:similarity-tests} and testing each quadrant in turn for evidence of non-uniformity, we confirm this suspicion. At the 5\% level, each  of the four largest clusters has one quadrant for which the null hypothesis of uniformity is not rejected. This leaves only the smallest cluster, containing just 37 points, 5\% of the total; we are forced to concede that not only is there is no evidence of a grid system common to the whole site, but that even at a regional level, no clear evidence of gridding can be found. % and at a 5% significance level, no less!

\begin{figure}[!ht]
\caption{Density-based clusters containing more than 25 points and found to have unimodal, reflectively symmetric distributions. Clusters are labelled according to the number of points they contain.} 
% Left-hand minipage
\begin{minipage}[t]{0.47\textwidth}
	\centering
	\subcaption{Non-uniform, reflectively symmetric large clusters identified}
	\label{fig:Cat-clusters}
 	\includegraphics[scale=0.35]{Catholme-clusters.pdf}
 \end{minipage}
 \hfill
 	%
 	% Right-hand minipage
	\begin{minipage}[t]{0.47\textwidth}
  		\centering
		\subcaption{Circular distribution of raw angles in each cluster. None of the plots suggest strong evidence of a 4-cyclic pattern.}
		\label{fig:clust-circ}
		\includegraphics[scale=0.45]{clust-circular.pdf}
    \end{minipage}
 \end{figure}
 
\subsection{Concluding remarks}
 
Although we have not detected any evidence of a dominant grid such as the one observed at Genlis, we should not be tempted to report that no underlying grid exists. \nb{Get Chris to help with phrasing on this bit...}


\end{document}
