\documentclass[../../ArchStats.tex]{subfiles}

\begin{document}
\section{Case Study 2: Catholme}
\label{sec:CS2}

A copy of the R code used to generate the plots and analysis can be found in Appendix \ref{app:CS2}.

\subsection{Point extraction}
\todo{Discuss difference in feature makeup between Catholme \& Genlis - results in a different set of steps. Less challenging to extract small, round post-hole features, but harder to extract pattern from the data}

\nb{Is 'sparse' the right word? Maybe non-dense}

 \begin{figure}[h!]
 \begin{minipage}[l]{0.47\textwidth}
 \caption{JPEG plan of Catholme site. The plan has minimal annotations and, having been taken directly from CAD data, is a strictly black-and-white image.}
 \centering
 \includegraphics[scale=0.2]{../../img/CS2-Catholme/Catholme-cropped.jpg}
 \end{minipage}
 \hfill
	\begin{minipage}[r]{0.47\textwidth}
	\caption{Final post-hole set extracted for Catholme site. Only dense features were retained, and isolated points or those not on a line have been removed.}
	\centering
	\includegraphics[scale=0.3]{../../img/CS2-Catholme/Catholme-postholes.pdf}
    \end{minipage}
 \end{figure}



\subsection{Fitting an angular distribution}

\subsubsection{Parameter estimation}
\subsubsection{Model selection}


\begin{figure}
\centering
\caption{Histograms of transformed angles $\boldsymbol{\theta}$, with densities of candidate models and kernel density estimate overlaid for reference. The  legend is common to both representations.}
\begin{subfigure}[t]{0.4\textwidth}
\caption{Circular histogram}
\centering
\includegraphics[scale=0.35]{../../img/CS2-Catholme/Q4-circ-plot.pdf}
\end{subfigure}
\begin{subfigure}[t]{0.4\textwidth}
\caption{Linear histogram, centred at sample mean}
\includegraphics[scale=0.3]{../../img/CS2-Catholme/Q4-linear-plot.pdf}
\end{subfigure}
\end{figure}

\subsection{Checking for gridding vs linearity}

\subsection{Checking for evidence of global gridding}

\subsection{Comparison of results using different point extraction methods}
ie. how robust is the process to different starting sets of post-holes?

\subsection{Evidence of consistent unit of measure?}

\end{document}