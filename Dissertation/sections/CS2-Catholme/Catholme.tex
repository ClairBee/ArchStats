\documentclass[../../ArchStats.tex]{subfiles}

\graphicspath{{/home/clair/Documents/ArchStats/Dissertation/sections/CS2-Catholme/img/}}

\begin{document}
\section{Case Study 2: Catholme}
\label{sec:CS2}

Our second case study is a plan of the dig site at Catholme. A more recent excavation than Genlis, the JPEG was generated directly from the CAD data rather than a scanned printed image; as a result, the image itself is much cleaner. However, the site is much larger and more complex than the one at Genlis, with post-hole structures and linear structures mixed together. In addition, the PEML team suspects that there may have been several phases of construction at this site, suggesting that there may not be a single dominant orientation. \nb{expand this a little}

A copy of the R code used to generate the plots and analysis can be found in Appendix \ref{app:CS2}.

 \begin{figure}[h!]
 \begin{minipage}[t]{0.47\textwidth}
 \caption{JPEG plan of Catholme site. The plan has minimal annotations and, having been taken directly from CAD data, is a strictly black-and-white image.}
 \centering
 \includegraphics[scale=0.2]{Catholme-cropped.jpg}
 \end{minipage}
 \hfill
	\begin{minipage}[t]{0.47\textwidth}
	\caption{Final post-hole set extracted for Catholme site. Only dense features were retained; remote points have been removed from the set.}
	\label{fig:Catholme-f-ext-postholes}
	\centering
	\includegraphics[scale=0.35]{Catholme-postholes.pdf}
    \end{minipage}
 \end{figure}
 
\subsection{Point extraction}
Identifying potential post-hole features is rather more straightforward here then in the first case study. We begin in the same way, by dividing dense features - in which more than 55\% of the shape's bounding square is coloured - from less dense ones, and excluding the latter from our set of post-hole candidates. In fact, this single step is sufficient to extract a useable set of post-hole features in this case; the few annotations that appear are in a finer font than those on the Genlis scan, and are immediately excluded, with the exception of the two `e's in the short perch scale label - and these two features are excluded when remote features are removed in the next step. For confirmation, features were excluded according to whether their shape was changed by a morphological closing - removing 28 features out of 979 remaining - and the function to fill dotted boundaries was run, removing a further 4 points. This trial serves as confirmation that the additional steps were neither necessary nor damaging in this case, removing only a handful of the available points. The final set of 915 post-holes identified is shown in Figure~\ref{fig:Catholme-f-ext-postholes}, with the 64 points excluded by distance filtering highlighted in red.






\subsection{Fitting an angular distribution}

The angles extracted from the final post-hole set are shown in Figure~\ref{fig:Catholme-angles}. The raw angles $\boldsymbol{\phi}$ in Figure~\ref{fig:Catholme-angles-raw} again show a 4-cyclic pattern, with the kernel density (smoothed using the same bandwidth, 30, as the Genlis data) having a rounded quadrilateral shape, suggesting that the post-holes do follow a perpendicular grid, rather than a linear one. However, the transformed angles in Figures~\ref{fig:Catholme-angles-trans-circ} and~\ref{fig:Catholme-angles-trans-linear} are less concentrated, having a smaller peak and a much higher proportion of the density lying far from the modal direction.

\begin{figure}[h!]
\centering
\caption{Histograms of raw angles $\boldsymbol{\phi}$ and transformed angles $\boldsymbol{\theta}$, with kernel density estimate and, where appropriate, densities of candidate models overlaid for reference. The  legend is common to both representations of $\boldsymbol{\theta}$.}
\label{fig:Catholme-angles}
%
\begin{subfigure}[t]{0.3\textwidth}
\caption{Raw angles $\boldsymbol{\phi}$\\ \textcolor{white}{spacer}}
\label{fig:Catholme-angles-raw}
\includegraphics[scale=0.35]{Q-circ-plot.pdf}
\end{subfigure}
%
\begin{subfigure}[t]{0.3\textwidth}
\centering
\caption{Transformed angles $\boldsymbol{\theta}$\\ \textcolor{white}{spacer}}
\label{fig:Catholme-angles-trans-circ}
\includegraphics[scale=0.35]{Q4-circ-plot.pdf}
\end{subfigure}
%
\begin{subfigure}[t]{0.3\textwidth}
\caption{Linear histogram of $\boldsymbol{\theta}$ \\(centred at $\bar{\theta}$)}
\label{fig:Catholme-angles-trans-linear}
\includegraphics[scale=0.3]{Q4-linear-plot.pdf}
\end{subfigure}
\end{figure}

Tests of uniformity of $\boldsymbol{\theta}$ confirm that the data is unimodal, with the Rayleigh test producing a result of $p = 0$, and Kuiper's and Watson's tests $p < 0.1$.  However, Pewsey's test of reflective symmetry gives $p = 0.019$, rejecting the null hypothesis that the data has reflective symmetry about some unspecified direction at the 2\% significance level, and suggesting that a symmetric distribution may not be appropriate here.


\subsubsection{Parameter estimation}

The bias-corrected estimate of $\bar{\beta}_2$ confirms the result of the test of reflective symmetry: the 95\% confidence interval does not contain 0. However, the degree of skew is only very small: the estimated value is -0.064, with the interval's upper limit just -0.0047. Furthermore, the circular median lies at 5.395, well within the confidence intervals for all of the estimates of $\mu$, suggesting that - although the skewness is non-zero - it need not prevent us altogether from attempting to fit a reflectively symmetric distribution. However, we should keep in mind that one plausible explanation for both the skewness and the low concentration of the data may be the presence of more than one underlying grid orientation in the post-holes. \nb{Possibly multiple grids?}

\begin{table}[!h]
\footnotesize
\centering
\caption{Bias-corrected summary statistics and MLE parameters for von Mises and Jones-Pewsey distributions, for the transformed angles $\boldsymbol{\theta}$ from the Catholme site. \nb{There is evidence that the data is more peaked than a von Mises distribution, so a Jones-Pewsey model may be more appropriate.}}
\label{tab:Catholme-statistics}
\begin{tabular}{c|cc|cc|cc}
\hline 
 & \multicolumn{2}{c|}{\textbf{Bias-corrected}} & \multicolumn{2}{c|}{\textbf{von Mises}} & \multicolumn{2}{c}{\textbf{Jones-Pewsey}} \\
\textbf{Parameter} & \textbf{Estimate} & \textbf{95\% CI} & \textbf{Estimate} & \textbf{95\% CI} & \textbf{Estimate} & \textbf{95\% CI} \\
\hline
$\mu$ & 5.500 & (5.300,  5.700) & 5.501 & (5.290, 5.712) & 5.360 & (5.220, 5.500) \\ 
$\rho$ & 0.213 & (0.167, 0.260) & 0.214 & (0.170, 0.258) &  0.247 & (0.198, 0.294) \\ 
$\kappa$ & 0.437 & (0.338, 0.538) & 0.439 & (0.344, 0.534) &  0.510 & (0.405, 0.616)\\ 
$\psi$ & - & - & - & - &  -3.465 & (-5.099, -1.831) \\ 
$\bar{\beta}_2$ & -0.064 & (-0.124, -0.005) & - & - & - & - \\ 
$\bar{\alpha}_2 $ & 0.125 & (0.070, 0.179) & - & - & - & - \\ 
\hline
\end{tabular}
\end{table}

The excess kurtosis $\bar{\alpha}_2 - \rho^4$ is estimated at 0.123, with a lower limit of 0.066, while the Jones-Pewsey $\psi$ parameter has been estimated to lie between -1.8 and -5.1, suggesting that, as previously, the Jones-Pewsey model may be more appropriate than the von Mises. Figure~\ref{fig:Catholme-angles-trans-linear} shows that, again, the von Mises distribution over-estimates the density at the shoulders of the distribution, while under-estimating the density at its peak; the maximum-likelihood Jones-Pewsey distribution, on the other hand, seems to match the kernel density well around the modal direction, although it is unable to reflect the smaller peak at $\nicefrac{5\pi}{2}$ ($\nicefrac{\pi}{2}$ in Figure~\ref{fig:Catholme-angles-trans-circ}), or the small dip at $\pi$.

\subsubsection{Model selection}

Goodness-of-fit tests using the circular probability integral transform method (Section~\ref{sec:unif-tests}) confirm this suspicion: while uniformity of the transformed Jones-Pewsey distribution is not rejected, with $p>0.1$ and $p>0.15$ from Kuiper's and Watson's tests respectively, the fit of the von Mises model is rejected at the 5\% level by both tests, with $0.025<p<0.05$ in both cases.

\todo{Test goodness of fit of each distribution. \nb{Look at PP \& QQ plots} as well as formal tests, to assess where the data fits well and where not.}

\subsection{Checking for gridding vs linearity}

\subsection{Checking for evidence of global gridding}

\nb{Distribution is quite dispersed compared to the Genlis site - maybe not particularly strictly aligned to a grid? At this point, need to try removing points of aligned features \& refitting}

\subsection{Comparison of results using different point extraction methods}
ie. how robust is the process to different starting sets of post-holes?

\subsection{Evidence of consistent unit of measure?}

\end{document}