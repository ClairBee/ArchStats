\documentclass[../../ArchStats.tex]{subfiles}

\begin{document}
\section{Extensions to the project}

Many approaches investigated in the course of developing the procedure outlined in this study did not find a place in the final method, but deserve closer consideration. A number of such extensions and refinements are proposed and briefly discussed below.

\subsection{Linear features}
This study has focussed solely on the remnants of post-holes, ignoring any larger linear features such as roads, walls, and trenches. However, these linear features represent the remains of structures that once formed part of the same  settlement as the post-holes, occupying the same space, and so may exhibit a similar orientation. Extending the analysis to include linear features would introduce some novel problems, chiefly in determining how best to represent a linear feature as a set of angles. One approach would be to simplify each linear feature into a number of short line segments, the angles of which are easily obtained.

The size of the set of angles obtained from each feature using this approach must be considered extremely carefully. A perfectly straight line running the length of the site might be simplified to a single long segment, resulting in only one angular measurement being taken: a set of angles not representative of the size of the feature.

A better method might be found based on investigation of the  distribution of the nearest-neighbour distances of the clustered points already found in the site, to determine the typical distance between adjacent post-holes. Measurements between points placed at intervals of this characteristic distance along linear features will still represent a simplified version of the angles along the  feature, but now the number of points will be similar to the number of points in a post-hole feature of comparable size. Using this approach, a linear feature can be represented by simulated post-holes, which can be added to the feature set along with the true post-holes and analysed alongside them. 

\subsection{Sites with multiple layers of gridding}
\label{sec:multiple-grids}
It is often the case that all of the buildings at a site were not built in a single phase of development, but may have been added to - even built over - in stages over a lengthy period of settlement. As a result, many sites may exhibit multiple grid orientations in varying regions, or even in the same space. A method to detect such mixtures of grids may prove insightful in cases where evidence of a single common grid orientation is not seen.

Here, the nearest-neighbour method proposed to determine the orientation of a point is unlikely to be sufficiently sensitive. Even in the relatively simple Genlis plan, points that appear to form a reasonably straight line have been assigned to different clusters by the winner-takes-all clustering shown in Figure~\ref{fig:Genlis-ph-clusts}. 

An alternative approach to determining the orientation of a point $i$ might begin by identifying the linear set of post-holes - if any - to which it belongs. For each site we should define a maximum radius within which to search for post-holes that form a line - a natural choice might be half of a short perch \cite{Blair2013}, or some quantile of the distribution of the nearest-neighbour distances for that site. For each post-hole $i$, we then measure the angle formed between it and every other pair of points within that specified radius.

Where the angle formed between $i$ and  another pair of nearby points is close to $\pi$ - that is, when the absolute difference between $\pi$ and the measured angle is less than some tolerance $\varepsilon$ - we have an $\varepsilon$-blunt triangle  \cite{Kendall1980}, containing three points that, when $\varepsilon$ is small (say $1^\circ$), are sufficiently closely aligned to be considered as points on a line with a small degree of perturbation. Searching the set of features for all $\varepsilon$-blunt triangles of nearby points will identify all such three-point line segments, which can be merged with adjacent segments to form sequences of closely-aligned points. For each group of post-holes connected by line segments, a least-squares method can be used to find a simple line of best fit to all of the points, giving a single angle that best describes the orientations of the points by minimising the perturbations.

This approach provides an objective method of identifying wall segments within a site, which will be effective even where multiple orientations are superposed, and so may able to detect sets of aligned points in clusters where the human eye could not; this in itself is a useful form of feature extraction. The orientations thus applied could then be used as the basis of the angular assessment outlined in this study.

\subsection{Assessment of common unit of measurement in grid}

Where strong evidence of a clear common orientation is detected, a further question of interest is that of whether the aligned points can be plausibly said to belong to a perpendicular grid of a particular unit. Evidence of a perpendicular grid based on a  unit such as the short perch \cite{Blair2013} would constitute extremely strong support for evidence of grid-planning.

Having obtained a set of points sharing a single perpendicular axis, between-wall (or between-feature)  might be obtained by measuring the distances between the orthogonal projections of points onto that axis. Frequency-based techniques, or the quantogram approach used in \cite{Kendall1974, Kendall2013}, can then be applied to test  for a common unit of measure.

This approach will depend on finding a large data set in which the angles show clear evidence of clustering and can be assigned to clusters representing walls, and it may often be the case that sufficiently accurate measurements cannot be obtained from the inferred clusters to perform a meaningful analysis.

%\subsection{Comparison of grid unit across multiple sites}
%Any evidence that a given site appears to be laid out on a grid of a particular length is of interest. However, if sufficient sites could be included in the study, a question of great interest to the PEML team would be whether multiple settlements are based on a grid of the same size.



%\subsection{Refinement of feature-extraction process}
%\todo{Possibly some sort of clustering algorithm, using features with similar characteristics (eg. similar dimensions, density etc) might be able to do a good job?}


%\subsection{PCA-based clustering of areas of map with similar orientation}
%Following methods proposed in \cite{Fisher1993} and \cite{Fisher1985}, a further approach to identifying which regions across a site share a similar orientation may be based on a principal coordinate analysis, as described in \cite{Gower1966}. \nb{Expand...}

\end{document}