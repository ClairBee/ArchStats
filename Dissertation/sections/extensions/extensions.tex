\documentclass[../../ArchStats.tex]{subfiles}

\begin{document}
\section{Extensions to the project}

\subsection{Linear features}
In this project, we have focussed entirely on the post-hole data available from the plans, ignoring any larger linear features such as roads, walls, and trenches. One extension would therefore be to develop a methodology to be applied to linear features as well as post-holes, to investigate whether this provides any evidence in support of the conclusions drawn based on the post-holes.

\subsection{Comparison of grid unit across multiple sites}
Any evidence that a given site appears to be laid out on a grid of a particular length is of interest. However, if sufficient sites could be included in the study, a question of great interest to the PEML team would be whether multiple settlements are based on a grid of the same size.

\subsection{Sites with multiple layers of gridding}
Although the PEML team are confident that the sites included in the study belong to the period of interest, it seems plausible that not all of the buildings at a site were laid out simultaneously, but may have been added to - even built over - in stages. As a result, many settlements may multiple grid orientations even in the same space. A method to detect such mixtures of grids may prove insightful in cases where evidence of a single common grid orientation cannot be found.

\end{document}