\documentclass[../../ArchStats.tex]{subfiles}

\begin{document}
\section{Extensions to the project}

Obviously, need to apply method to more sites: may come up with further adjustments \& refinements as a result.

\subsection{Explore relationship between Jones-Pewsey and uniform-von Mises mixture model?}

\subsection{Spatial autocorrelation of angles to get a more rigorous measure?}

\subsection{Sensitivity: how big a difference does there need to be between two directions before we can detect multiple axes?}

\subsection{Improvements to EM algorithm}
Rather than generating soft assignment and using to create a hard clustering, could generate hard assignment throughout. Don't know what this would do to the estimated parameters though.

\subsection{Establish expected degree of concentration for post-holes on a wall}
If there is a standard degree of concentration for post-holes along a single line, we can use this to reverse-engineer the `expected' $\kappa$, since a von Mises with $\kappa > 2$ can be approximated by a Normal with $\sigma^2 = \nicefrac{1}{\kappa}$. Rather than estimating $\kappa$ from the data, we could then place a distribution with said $\kappa$ at the mean direction and use to cluster the angles to identify commonly-oriented structures. Also by removing that cluster, may be able to identify multiple further orientations.

\subsection{Linear features}
In this project, we have focussed entirely on the post-hole data available from the plans, ignoring any larger linear features such as roads, walls, and trenches. One extension would therefore be to develop a methodology to be applied to linear features as well as post-holes, to investigate whether this provides any evidence in support of the conclusions drawn based on the post-holes.

\subsection{Assessment of common unit of measurement in grid}
We've found evidence that points are aligned to one another in reasonably remote areas of the grid. If, furthermore, those points are a fixed distance apart, we have extremely strong evidence of gridding. Need to extract distances \& assess whether there is evidence of a common unit of measure being applied: particularly between buildings/structures.


\subsection{Comparison of grid unit across multiple sites}
Any evidence that a given site appears to be laid out on a grid of a particular length is of interest. However, if sufficient sites could be included in the study, a question of great interest to the PEML team would be whether multiple settlements are based on a grid of the same size.

\subsection{Sites with multiple layers of gridding}
Although the PEML team are confident that the sites included in the study belong to the period of interest, it seems plausible that not all of the buildings at a site were laid out simultaneously, but may have been added to - even built over - in stages. As a result, many settlements may multiple grid orientations even in the same space. A method to detect such mixtures of grids may prove insightful in cases where evidence of a single common grid orientation cannot be found.

%\subsection{Refinement of feature-extraction process}
%\todo{Possibly some sort of clustering algorithm, using features with similar characteristics (eg. similar dimensions, density etc) might be able to do a good job?}


%\subsection{PCA-based clustering of areas of map with similar orientation}
%Following methods proposed in \cite{Fisher1993} and \cite{Fisher1985}, a further approach to identifying which regions across a site share a similar orientation may be based on a principal coordinate analysis, as described in \cite{Gower1966}. \nb{Expand...}

\end{document}