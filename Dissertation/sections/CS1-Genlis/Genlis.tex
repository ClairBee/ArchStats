\documentclass[../../ArchStats.tex]{subfiles}

\begin{document}

\section{Case Study 1: Genlis}
\label{sec:CS1}

Our first case study is a site Genlis site in northern France \nb{? - get a brief description of site from Chris S).}

A copy of the R code used to generate the plots and analysis can be found in Appendix \ref{app:CS1}.

\subsection{Description of data set}

\begin{figure}[!h]
\centering
\caption{JPEG scan of Genlis site map. Some smudges and print from the reverse of the scanned page are visible and must be removed before angles can be extracted, along with all annotations and the broken line denoting the site boundary.}
\includegraphics[scale=0.3]{../../img/CS1-Genlis/Genlis-cropped.jpg}
\end{figure}

\nb{look at Fisher \cite{Fisher1985} for how to introduce \& describe a data set.}

\subsection{Converting JPEG to points}

Following the procedure given in \ref{sec:points-to-JPEG}, we begin by using the scale marker to rescale the raster's $x$ and $y$ axes from the arbitrary values generated when importing the data to a more realistic scale in metres. Manual identification of the N-S marker tells us that due north lies at an angle of 1.97 radians ($113^\circ$) from the horizontal origin against which the rest of the angles are measured.

\todo{Stop splitting out linear features at this point - can be part of an extension later}

\nb{Final decision on approach will probably need to be based on Catholme...}
 
\nb{Need to settle on a final colour scheme...}

\begin{figure}
\centering
\caption{Extracting features from the Genlis site plan using a morphological closing. The coordinate system has been rescaled to reflect the true dimensions and is given in metres.}
%
\begin{subfigure}[b]{0.46\textwidth}
\caption{Sparse features identified and classified as annotations}
\centering
\includegraphics[scale=0.3]{../../img/CS1-Genlis/Genlis-sparse.pdf}
\end{subfigure}
%
\begin{subfigure}[b]{0.46\textwidth}
\caption{Further annotations identified by comparing feature size before and after morphological closing}
\centering
\includegraphics[scale=0.3]{../../img/CS1-Genlis/Genlis-after-closing.pdf}
\end{subfigure}

\vspace{10pt}
%
\begin{subfigure}[b]{0.46\textwidth}
\caption{Points on broken boundary line identified by extending linear features}
\centering
\includegraphics[scale=0.3]{../../img/CS1-Genlis/Genlis-boundary-filled.pdf}
\end{subfigure}
%
\begin{subfigure}[b]{0.46\textwidth}
\caption{Final post-hole set selected}
\centering
\includegraphics[scale=0.3]{../../img/CS1-Genlis/Genlis-1-postholes.pdf}
\end{subfigure}
%
\end{figure}






\subsection{Fitting the global distribution}






\subsubsection{Parameter estimation}

\begin{table}[!h]
\footnotesize
\caption{Bias-corrected summary statistics and MLE parameters for the von Mises and Jones-Pewsey distributions, for the Genlis data set. \nb{How to display row headers properly? May need manual correction...}}%\begingroup\catcode"=9
\csvreader[tabular=c|c c|c c|c c,
    table head= \hline \textbf{Parameter} & \textbf{BC Estimate} & \textbf{CI} & \textbf{vM Estimate} & \textbf{CI} & \textbf{JP Estimate} & \textbf{CI} \\\hline,
    table foot=\bottomrule, head to column names]
    {./csv/Genlis-simulated-ests.csv}
    {1=\param, 2 = \BCest, 3=\BClower, 4 = \BCupper, 5 = \VMest, 6 = \VMlower, 7 = \VMupper, 8 = \JPest, 9 = \JPlower, 10 = \JPupper}
    {\param & \BCest & (\BClower, \BCupper) &\VMest & (\VMlower, \VMupper) & \JPest & (\JPlower, \JPupper)}
%    \endgroup
\end{table}

\subsubsection{Model selection}



\begin{figure}
\centering
\caption{Transformed angles $\boldsymbol{\theta}$ with densities of candidate models}
\begin{subfigure}[t]{0.45\textwidth}
\caption{Circular plot}
\includegraphics[scale=0.45]{./img/genlis-circ-plot.pdf}
\end{subfigure}
\begin{subfigure}[t]{0.45\textwidth}
\caption{Linear plot}
\includegraphics[scale=0.45]{./img/genlis-linear-plot.pdf}
\end{subfigure}
\end{figure}


\subsection{Gridding vs linearity}



\subsection{Global vs local gridding}
\todo{Compare different approaches to splitting data: via midpoints or via clustering (if clustering, use same method as Catholme. Smaller $\varepsilon$ divided that large cluster into a set of smaller sub-features, which showed evidence of a common distribution but could also be considered to be part of the same feature: so common distribution is not surprising. Should therefore err on side of a large $\varepsilon$ rather than a smaller, to avoid treating parts of a larger composite feature as if they are separate parts.}

\subsection{Comparison of results using different point extraction methods}
ie. how robust is the process to different starting sets of post-holes?

\subsection{Comparison of points with perturbed data}


\subsection{Evidence of consistent unit of measure?}



\end{document}