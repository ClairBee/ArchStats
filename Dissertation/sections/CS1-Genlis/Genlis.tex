\documentclass[../../ArchStats.tex]{subfiles}

\begin{document}

\section{Case Study 1: Genlis}
\label{sec:CS1}

Our first case study is the Genlis site in northern France \nb{? - get a brief description of site from Chris S).}

A copy of the R code used to generate the plots and analysis can be found in Appendix \ref{app:CS1}.

\subsection{Description of data set}

\subsection{Converting JPEG to points}

Following the procedure given in \ref{sec:points-to-JPEG}, we begin by using the scale marker to rescale the raster's $x$ and $y$ axes from the arbitrary values generated when importing the data to a more realistic scale in metres. Manual identification of the N-S marker tells us that due north lies at an angle of 0.4040071 radians ($23^\circ$) from the \nb{vertical origin or horizontal origin?} against which the rest of the angles are measured; we will use this to adjust the values of the measured angles when reporting.

\nb{Need to sort out plotting issues...}
\begin{figure}
\centering
\caption{Stages of pre-processing of the Genlis site plan}
\begin{subfigure}[t]{0.45\textwidth}
\caption{Before processing}
\includegraphics[scale=0.2]{./img/Genlis-cropped.jpg}
\end{subfigure}
\begin{subfigure}[t]{0.45\textwidth}
\caption{Features classified as post-holes (black) or annotations (light blue)}
\includegraphics[scale=0.5]{./img/genlis-postholes.pdf}
\end{subfigure}
\end{figure}

\subsection{Fitting the global distribution}





\subsubsection{Parameter estimation}

\begin{table}[!h]
\footnotesize
\caption{Bias-corrected summary statistics and MLE parameters for the von Mises and Jones-Pewsey distributions, for the Genlis data set. \nb{How to display row headers properly? May need manual correction...}}%\begingroup\catcode"=9
\csvreader[tabular=c|c c|c c|c c,
    table head= \hline \textbf{Parameter} & \textbf{BC Estimate} & \textbf{CI} & \textbf{vM Estimate} & \textbf{CI} & \textbf{JP Estimate} & \textbf{CI} \\\hline,
    table foot=\bottomrule, head to column names]
    {./csv/Genlis-simulated-ests.csv}
    {1=\param, 2 = \BCest, 3=\BClower, 4 = \BCupper, 5 = \VMest, 6 = \VMlower, 7 = \VMupper, 8 = \JPest, 9 = \JPlower, 10 = \JPupper}
    {\param & \BCest & (\BClower, \BCupper) &\VMest & (\VMlower, \VMupper) & \JPest & (\JPlower, \JPupper)}
%    \endgroup
\end{table}

\subsubsection{Model selection}



\begin{figure}
\centering
\caption{Transformed angles $\boldsymbol{\theta}$ with densities of candidate models}
\begin{subfigure}[t]{0.45\textwidth}
\caption{Circular plot}
\includegraphics[scale=0.45]{./img/genlis-circ-plot.pdf}
\end{subfigure}
\begin{subfigure}[t]{0.45\textwidth}
\caption{Linear plot}
\includegraphics[scale=0.45]{./img/genlis-linear-plot.pdf}
\end{subfigure}
\end{figure}


\subsection{Gridding vs linearity}



\subsection{Global vs local gridding}
\todo{Compare different approaches to splitting data: via midpoints or via clustering (if clustering, use same method as Catholme. Smaller $\varepsilon$ divided that large cluster into a set of smaller sub-features, which showed evidence of a common distribution but could also be considered to be part of the same feature: so common distribution is not surprising. Should therefore err on side of a large $\varepsilon$ rather than a smaller, to avoid treating parts of a larger composite feature as if they are separate parts.}

\subsection{Comparison of results using different point extraction methods}
ie. how robust is the process to different starting sets of post-holes?

\subsection{Comparison of points with perturbed data}


\subsection{Evidence of consistent unit of measure?}



\end{document}