\documentclass[../../ArchStats.tex]{subfiles}

\begin{document}

\section{Case Study 1: Genlis}
\label{sec:CS1}

Our first case study is the Genlis site in northern France \nb{? - get a brief description of site from Chris S).}

The R code used to generate the plots and analysis can be found in Appendix\nb{REF}

\subsection{Converting JPEG to points}

Following the procedure given in \ref{sec:points-to-JPEG}, we begin by using the scale marker to rescale the raster's $x$ and $y$ axes from the arbitrary values generated when importing the data to a more realistic scale in metres. Manual identification of the N-S marker tells us that due north lies at an angle of 0.4040071 radians ($23^\circ$) from the \nb{vertical origin or horizontal origin?} against which the rest of the angles are measured; we will use this to adjust the values of the measured angles when reporting.

\nb{Need to sort out plotting issues...}
\begin{figure}
\centering
\caption{Stages of pre-processing of the Genlis site plan}
\begin{subfigure}[t]{0.45\textwidth}
\caption{Before processing}
\includegraphics[scale=0.2]{./img/Genlis-cropped.jpg}
\end{subfigure}
\begin{subfigure}[t]{0.45\textwidth}
\caption{Features classified as post-holes (black) or annotations (light blue)}
\includegraphics[scale=0.5]{./img/genlis-postholes.pdf}
\end{subfigure}
\end{figure}

\subsection{Fitting the global distribution}





\subsubsection{Parameter estimation}

\begin{table}[!h]
\footnotesize
\caption{Bias-corrected summary statistics and MLE parameters for the von Mises and Jones-Pewsey distributions, for the Genlis data set. \nb{How to display row headers properly? May need manual correction...}}%\begingroup\catcode"=9
\csvreader[tabular=c|c c|c c|c c,
    table head= \hline \textbf{Parameter} & \textbf{BC Estimate} & \textbf{CI} & \textbf{vM Estimate} & \textbf{CI} & \textbf{JP Estimate} & \textbf{CI} \\\hline,
    table foot=\bottomrule, head to column names]
    {./csv/Genlis-simulated-ests.csv}
    {1=\param, 2 = \BCest, 3=\BClower, 4 = \BCupper, 5 = \VMest, 6 = \VMlower, 7 = \VMupper, 8 = \JPest, 9 = \JPlower, 10 = \JPupper}
    {\param & \BCest & (\BClower, \BCupper) &\VMest & (\VMlower, \VMupper) & \JPest & (\JPlower, \JPupper)}
%    \endgroup
\end{table}

\subsubsection{Model selection}

\begin{verbatim}
> rayleigh.test(q.4); kuiper.test(q.4); watson.test(q.4)
       Rayleigh Test of Uniformity 
       General Unimodal Alternative 
			P-value:  0 
       Kuiper's Test of Uniformity 
 			P-value < 0.01 
        Watson's Test for Circular Uniformity 
 			P-value < 0.01 
> r.symm.test.stat(q.4); r.symm.test.boot(q.4, B = 999)
			$p.val
			[1] 0.7382032
\end{verbatim}

\begin{figure}
\centering
\caption{Transformed angles $\boldsymbol{\theta}$ with densities of candidate models}
\begin{subfigure}[t]{0.45\textwidth}
\caption{Circular plot}
\includegraphics[scale=0.45]{./img/genlis-circ-plot.pdf}
\end{subfigure}
\begin{subfigure}[t]{0.45\textwidth}
\caption{Linear plot}
\includegraphics[scale=0.45]{./img/genlis-linear-plot.pdf}
\end{subfigure}
\end{figure}

\begin{verbatim}
> vM.GoF(q.4, vm.mle$mu, vm.mle$kappa)
       Watson's Test for the von Mises Distribution 
			P-value < 0.01 
       Kuiper's Test of Uniformity 
 			0.01 < P-value < 0.025 
        Watson's Test for Circular Uniformity 
 			0.01 < P-value < 0.025 
        Rayleigh Test of Uniformity 
       General Unimodal Alternative 
			P-value:  0.2584 

> JP.GoF(q.4, jp.mle$mu, jp.mle$kappa, jp.mle$psi)
       Kuiper's Test of Uniformity 
 			P-value > 0.15 
        Watson's Test for Circular Uniformity 
 			P-value > 0.10 
        Rayleigh Test of Uniformity 
       General Unimodal Alternative 
			P-value:  0.9831 

      von Mises Jones-Pewsey
AICc      7.724      856.877
\end{verbatim}

\subsection{Gridding vs linearity}

\begin{verbatim}
> watson.common.mean.test(q.samples)
			$p.val				$disp.ratio
			[1] 0.8156225		[1] 1.115793
> wallraff.concentration.test(q.samples)
			$p.val
			[1] 0.5092869
> mww.common.dist.LS(cs.unif.scores(q.samples), q.sizes)
			$p.val
			[1] 0.8055934
\end{verbatim}

\subsection{Global vs local gridding}
\todo{Compare different approaches to splitting data: via midpoints or via clustering (if clustering, use same method as Catholme. Smaller $\varepsilon$ divided that large cluster into a set of smaller sub-features, which showed evidence of a common distribution but could also be considered to be part of the same feature: so common distribution is not surprising. Should therefore err on side of a large $\varepsilon$ rather than a smaller, to avoid treating parts of a larger composite feature as if they are separate parts.}

\subsection{Comparison of results using different point extraction methods}
ie. how robust is the process to different starting sets of post-holes?

\subsection{Comparison of points with perturbed data}


\subsection{Evidence of consistent unit of measure?}

\section{Temp: Boxplots of feature dimensions at each stage of data cleaning}
\begin{figure}
\centering
\begin{subfigure}[t]{0.3\textwidth}
\caption{Initial feature set}
\includegraphics[scale=0.25]{./img/genlis-dims-1-all-but-scales.pdf}
\end{subfigure}
%
\begin{subfigure}[t]{0.3\textwidth}
\caption{Sparse features removed}
\includegraphics[scale=0.25]{./img/genlis-dims-2-sparse-removed.pdf}
\end{subfigure}
%
\begin{subfigure}[t]{0.3\textwidth}
\caption{Horizontal strings removed}
\includegraphics[scale=0.25]{./img/genlis-dims-3-annotations-removed.pdf}
\end{subfigure}

\begin{subfigure}[t]{0.3\textwidth}
\caption{Tall features removed}
\includegraphics[scale=0.25]{./img/genlis-dims-4-tall-features-removed.pdf}
\end{subfigure}
%
\begin{subfigure}[t]{0.3\textwidth}
\caption{Annotations filled in}
\includegraphics[scale=0.25]{./img/genlis-dims-5-inter-annotations-removed.pdf}
\end{subfigure}
\end{figure}

\begin{figure}
\centering
\caption{Dims of full feature set; sparse features removed; horizontal strings removed; tall features removed; annotations filled in.}
\begin{subfigure}[b]{0.4\textwidth}
\caption{Height}
\includegraphics[scale=0.25]{./img/genlis-PH-height.pdf}
\end{subfigure}
%
\begin{subfigure}[b]{0.4\textwidth}
\caption{Width}
\includegraphics[scale=0.25]{./img/genlis-PH-width.pdf}
\end{subfigure}

\begin{subfigure}[b]{0.4\textwidth}
\caption{Density (\% black cells within bounding square)}
\includegraphics[scale=0.25]{./img/genlis-PH-density.pdf}
\end{subfigure}
%
\begin{subfigure}[b]{0.4\textwidth}
\caption{Ratio of longest to shortest edge}
\includegraphics[scale=0.25]{./img/genlis-PH-ratio.pdf}
\end{subfigure}
\end{figure}

\end{document}