\documentclass[../../ArchStats.tex]{subfiles}

\graphicspath{{/home/clair/Documents/ArchStats/Dissertation/sections/CS1-Genlis/img/}}

\begin{document}

\section{Case Study 1: Genlis}
\label{sec:CS1}

Our first case study is an excavation site in Genlis, in northern France (Figure~\ref{fig:Genlis-JPEG}). Of the site plans provided by the PEML team, this is among the smallest, with its post-hole structures generally well separated from the annotations and larger features; to the human eye, there certainly appear to sets of post-holes at right-angles to one another. For this reason we have chosen to focus first on this relatively straightforward case, before moving on to consider a more complex site in the next section.

A copy of the R code used to generate the plots and analysis can be found in Appendix \ref{app:CS1}.

\begin{figure}[!h]
\label{fig:Genlis-JPEG}
\centering
\caption{JPEG scan of Genlis site map. Some smudges and print from the reverse of the scanned page are visible and must be removed before angles can be extracted, along with all annotations and the broken line denoting the site boundary.}
\includegraphics[scale=0.3]{Genlis-cropped.jpg}
\end{figure}

%\nb{look at Fisher \cite{Fisher1985} for how to introduce \& describe a data set.}

\subsection{Converting JPEG to points}

Following the procedure given in \ref{sec:points-to-JPEG}, we import the data and convert it to a black-and-white raster object. A threshold value of 0.2 is high enough to transform the plan's background to solid white, even where text from the reverse of the scanned page has been picked up; however, a number of short smudges have been converted to black pixels.

The scale marker is used to rescale the raster's $x$ and $y$ coordinates to a more realistic scale; any measurements given in relation to the plan will therefore now be given in metres. Manual identification of the group of pixels representing the N-S marker tells us that due north lies at an angle of 1.97 radians ($113^\circ$) from the horizontal origin against which the rest of the angles are measured.

As a first step, we exclude any `sparse' features, defined as in section~\ref{sec:excl-sparse} as any feature that covers less than 55\% of the pixels in its bounding square. As Figure~\ref{fig:Genlis-f-ext-sparse-removed} shows, this process has removed most - but not all - of the numerical annotations; all of the larger features represented by lines; and the lines of the site boundary. However, we are left with a handful of letters and numbers, a few larger circular features represented by shaded circles, and the dots of the boundary line.

Removing any of those shapes whose sizes are changed by morphological closing, as in section \ref{sec:closing}, results in the changes highlighted in Figure~\ref{fig:Genlis-f-ext-closed}. Of the annotations, a single letter `t' remains in the heading at the top of the map, while a single number 4 remains at around (53, 8), just above the lower site boundary; these features are dense and have no holes, making them morphologically very similar to post-holes. If a number of adjacent annotations remained, we would risk introducing a spurious \nb{?} cluster of orientations into our angular measurements, so we would need to run a further filter to remove the points; however, both of these are solitary points, adding a tiny amount of noise to our data set, and so we can allow them to remain, rather than risk removing useful points by over-filtering the data.

As a final step, we need to distinguish genuine post-hole features from points on the broken boundary line; to include these would introduce a large number of points with similar orientations, which could be genuinely damaging to the analysis. We therefore use the exhaustive-search method outlined in Section~\ref{sec:site-boundaries}, which successfully identifies the majority of points, as shown in Figure~\ref{fig:Genlis-f-ext-boundary}. The final set of post-holes identified is represented more clearly in Figure~\ref{fig:Genlis-f-ext-postholes}; although a small number of points on the boundary have been missed by the procedure, we can see that again, there are no sequences of more than two such points, so the effect is only to introduce a small amount of additional noise into the data set, and we need not be too concerned. In particular, after removing particularly remote points (highlighted in red in Figure~\ref{fig:Genlis-f-ext-postholes}), several of the erroneous annotation marks have been removed anyway, leaving a set of 243 post-holes to be analyzed.


\begin{figure}[h!]
\label{fig:Genlis-f-ext}
\centering
\caption{Stages of post-hole feature identification process for the Genlis site plan. \nb{Decide on nearest-neighbour filter based on Catholme results.}}
%
\begin{subfigure}[b]{0.46\textwidth}
\caption{Sparse features removed}
\label{fig:Genlis-f-ext-sparse-removed}
\centering
\includegraphics[scale=0.3]{Genlis-sparse.pdf}
\end{subfigure}
%
\begin{subfigure}[b]{0.46\textwidth}
\caption{Further annotations identified by comparing feature size before and after morphological closing}
\label{fig:Genlis-f-ext-closed}
\centering
\includegraphics[scale=0.3]{Genlis-after-closing.pdf}
\end{subfigure}

\vspace{10pt}
%
\begin{subfigure}[b]{0.46\textwidth}
\caption{Points on broken boundary line identified by extending linear features}
\label{fig:Genlis-f-ext-boundary}
\centering
\includegraphics[scale=0.3]{Genlis-boundary-filled.pdf}
\end{subfigure}
%
\begin{subfigure}[b]{0.46\textwidth}
\caption{Final post-hole set selected}
\label{fig:Genlis-f-ext-postholes}
\centering
\includegraphics[scale=0.3]{Genlis-1-postholes.pdf}
\end{subfigure}
%
\end{figure}






\subsection{Fitting the global distribution}

\begin{verbatim}
rayleigh.test(q.4)                  # p = 0
kuiper.test(q.4)                    # p < 0.01
watson.test(q.4)                    # p < 0.01

r.symm.test.stat(q.4)               # p = 0.959
r.symm.test.boot(q.4, B = 999)      # p = 0.949
\end{verbatim}




\subsubsection{Parameter estimation}

\begin{table}[!h]
\footnotesize
\centering
\caption{Bias-corrected summary statistics and MLE parameters for von Mises and Jones-Pewsey distributions, for the transformed angles $\boldsymbol{\theta}$ from the Genlis site.}
\label{tab:sim-statistics}
\begin{tabular}{c|cc|cc|cc}
\hline 
 & \multicolumn{2}{c|}{\textbf{Bias-corrected}} & \multicolumn{2}{c|}{\textbf{von Mises}} & \multicolumn{2}{c}{\textbf{Jones-Pewsey}} \\
\textbf{Parameter} & \textbf{Estimate} & \textbf{95\% CI} & \textbf{Estimate} & \textbf{95\% CI} & \textbf{Estimate} & \textbf{95\% CI} \\
\hline
$\mu$ & 3.217 & (2.964, 3.471) & 3.217 & (2.925, 3.510) & 3.231 & (3.047, 3.416) \\ 
$\rho$ & 0.390 & (0.270, 0.511) & 0.393 & (0.280, 0.491) & 0.393 & (0.280, 0.491) \\ 
$\kappa$ & 0.848 & (0.562, 1.183) & 0.856 & (0.583, 1.130) & 0.955 & (0.707, 1.204) \\ 
$\psi$ & - & - & - & - & -1.847 & (-2.873, -0.820) \\ 
$\bar{\beta}_2$ & 0.004 & (-0.163, 0.171) & - & - & - & - \\ 
$\bar{\alpha}_2 $ & 0.316 & (0.205, 0.427) & - & - & - & - \\ 
\hline
\end{tabular}
\end{table}


\subsubsection{Model selection}



\begin{figure}[h!]
\label{fig:Genlis-angles}
\centering
\caption{Histograms of raw angles $\boldsymbol{\phi}$ and transformed angles $\boldsymbol{\theta}$, with kernel density estimate and, where appropriate, densities of candidate models overlaid for reference. The  legend is common to both representations of $\boldsymbol{\theta}$.}
%
\begin{subfigure}[t]{0.3\textwidth}
\label{fig:Genlis-angles-raw}
\caption{Raw angles $\boldsymbol{\phi}$}
\includegraphics[scale=0.35]{Q-circ-plot.pdf}
\end{subfigure}
%
\begin{subfigure}[t]{0.3\textwidth}
\label{fig:Genlis-angles-trans-circ}
\caption{Transformed angles $\boldsymbol{\theta}$}
\includegraphics[scale=0.35]{Q4-circ-plot.pdf}
\end{subfigure}
%
\begin{subfigure}[t]{0.3\textwidth}
\label{fig:Genlis-angles-trans-linear}
\caption{Linear histogram of $\boldsymbol{\theta}$}
\includegraphics[scale=0.3]{Q4-linear-plot.pdf}
\end{subfigure}
\end{figure}


\begin{figure}[h!]
\label{fig:Genlis-prob-plots}
\centering
\caption{Probability plots of the fitted candidate distributions and their residuals.}
%
\begin{subfigure}[t]{0.24\textwidth}
\caption{P-P plot}
\includegraphics[scale=0.2]{PP-plot.pdf}
\end{subfigure}
%
\begin{subfigure}[t]{0.24\textwidth}
\caption{P-P residuals}
\includegraphics[scale=0.2]{PP-residuals.pdf}
\end{subfigure}
%
\begin{subfigure}[t]{0.24\textwidth}
\caption{Q-Q plot}
\includegraphics[scale=0.2]{QQ-plot.pdf}
\end{subfigure}
%
\begin{subfigure}[t]{0.24\textwidth}
\caption{Q-Q residuals}
\includegraphics[scale=0.2]{QQ-residuals.pdf}
\end{subfigure}
%
\end{figure}

\subsection{Gridding vs linearity}

\begin{verbatim}
watson.common.mean.test(q.samples)                          # p = 0.76
wallraff.concentration.test(q.samples)                      # p = 0.77
mww.common.dist.LS(cs.unif.scores(q.samples), q.sizes)      # p = 0.58
watson.two.test(q.4.a, q.4.b)                               # p > 0.10
watson.two.test.rand(q.4.a, q.4.b, NR = 999)                # p = 0.68
\end{verbatim}


\subsection{Global vs local gridding}
\todo{Compare different approaches to splitting data: via midpoints or via clustering (if clustering, use same method as Catholme. Smaller $\varepsilon$ divided that large cluster into a set of smaller sub-features, which showed evidence of a common distribution but could also be considered to be part of the same feature: so common distribution is not surprising. Should therefore err on side of a large $\varepsilon$ rather than a smaller, to avoid treating parts of a larger composite feature as if they are separate parts.}

\nb{Describe the `true' underlying distribution using a mixture model obtained by EM, if possible}

\subsection{Comparison of results using different point extraction methods}
ie. how robust is the process to different starting sets of post-holes?

Data was cleaned by removing remote points \& those not appearing to be part of a regular feature. What would have been the outcome if we hadn't removed those points?

\todo{Give table of comparison}
\todo{Fit final model}

Tried using raster threshold 0.3 instead of 0.2. Picked up 268 points instead of 267.

\nb{Effect of removing points that don't appear to be part of a regular feature?} Massively reduces size of data set.

\subsection{Evidence of consistent unit of measure?}



\end{document}