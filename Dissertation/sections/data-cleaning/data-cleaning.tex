\documentclass[../../ArchStats.tex]{subfiles}

\graphicspath{{/home/clair/Documents/ArchStats/Dissertation/sections/data-cleaning/img/}}

\begin{document}

% Was: 5 pages, 3708 words.
% Removed sections on data cleaning methods not used: 
% Added section on morphological closing: 

 
\section{Data cleaning}
\label{sec:data-cleaning}
Plans of dig sites of particular interest were provided by the PEML team in JPEG format; before any analysis can be carried out on the angular relationships between the post-holes and other features, the images must be converted into a set of points representing the locations of the features of interest.

There is no standard format for the printing and publication of archaeological plans - line weights, fonts, background colours and line types may all differ from one plan to another - so it is impossible to specify a precise set of steps that will work well for every image. In subsection \ref{sec:points-to-JPEG} we will describe a process that can be applied with reasonable results to site maps of a particular style, while in \ref{sec:alternative-techniques}, a number of more specific techniques are described, which can be applied to obtain different feature classifications. In section \ref{sec:posts-to-angles}, the extraction of an appropriate set of angles to represent the orientation of the post-holes is discussed.



\subsection{Outline procedure to identify post-holes from a JPEG image}
\label{sec:points-to-JPEG}

We assume (hopefully not unreasonably) that all maps will be scanned in such a way that text and other annotations, such as the legend, are more or less horizontal. Image processing prior to loading the image into R should be kept to a minimum, but cropping out figure labels and borders that are clearly external to the site plan is a useful step. Where possible, the scale key and N-S axis arrow should be kept on the scanned image, so that all measurements taken from the data can be related to the site's true scale and orientation.

Our initial focus is on the orientation of post-hole features; these are generally represented as solid points, and are smaller than the text used to annotate the plan. The procedure given will assume that this is the case; if the post-holes are represented as outlines or are larger than the text, or if linear features are to be extracted, then a different approach will be required. In all cases, a certain amount of trial and error is to be expected to establish the most appropriate procedure and parameters for a particular site plan.

One of the most important considerations when separating post-holes from other features must be to ensure that no regular annotations (such as text or site boundaries) are picked up and treated as part of the post-hole set. This would introduce lines of points along a shared orientation, which are likely to have more of an adverse effect on our analysis of the angles between the post-holes than would excluding a small number of post-holes or including one or two accidentally-introduced, but randomly distributed points.

Some subjective judgement is necessary to assess when we have reached an `adequate' level of separation between post-hole and non-post-hole features. \nb{How can we assess this? Use boxplots of the dimensions of the features? - if all similar size \& shape, this may help. But will always depend on the site.}



\subsubsection{Extract features from JPEG}

The JPEG image is loaded and immediately converted to a raster object, which assigns a numerical value to  each pixel in the image. At this point, with no other information available from the JPEG file, the $x$-coordinates are set by default from 0 to 1 - with the resolution determined by the number of pixels in a row - and the $y$-coordinates are scaled in such a way that the aspect ratio of the image is maintained.

The raster values initially encode a full-colour image - although most of these colours will be shades of grey if the JPEG was of a black-and-white map - and the values must be binarized, replacing those below a certain threshold $t$ with 0 (a white pixel) and those above with 1 (a black pixel). For most black-and-white images, very similar results will generally be obtained with $0.1 < t < 0.9$; however, for full-colour images, a high value of $t$ might be necessary to avoid converting shaded areas to solid black, while for images with particularly fine or faint lines, a low value of $t$ will be more useful, to avoid  breaking those lines into small fragments that may resemble post-holes in size and shape. For most black-and-white images, a relatively low threshold of around $t = 0.2$ should be preferred; while small smudges and other `noise' features are more likely be picked up by such a low threshold, it is generally less potentially problematic to permit a few erroneous but randomly-distributed points such as these, rather than risk wrongly identifying groups of fragments of letters or site boundaries - which are more likely to appear in regular lines, and so to interfere with the angular analysis - as small, post-hole-type features.

Clumps of adjacent black pixels are identified as individual features of the map and numbered for reference; diagonally adjacent pixels are included in this definition for the same reason that a lower $t$-value is recommended for lighter images. The raster object containing the feature numbers will henceforth be referred to as the feature raster. Individual features can now be classified according to their shape, size, and proportions.

\subsubsection{Rescale the plan}
\label{sec:rescale}

As long as the map's annotations are horizontally aligned, the map's scale marker can generally be identified by designing a focal window that scores highly only when long, horizontal lines of black pixels are detected within it, and passing the window over the feature raster; the largest such horizontal feature is will generally be the scale marker. On user confirmation of the true distance represented by the feature thus identified, the feature raster's $x$ and $y$ coordinates can be rescaled, allowing analysis of the map to be carried out in approximately realistic units rather than the arbitrary scale that would otherwise be used.

The accuracy of the revised scale will depend on both the scale and accuracy of the original map and the resolution of the JPEG image, so detailed conclusions about distances should be checked against more accurate measurements and revised accordingly. However, since our main interest is in finding well-separated regions of a site that share a similar orientation, the accuracy achieved should be sufficient.

It will generally be assumed that all maps have been converted to their `true' scale in this way, unless stated otherwise. In practice, if rescaling of the map is not possible for some reason, all of the techniques listed may still be used, but estimation of distance-based parameters and assessments of the degree of separation between features must be made in arbitrary units.

Ideally, the direction of the N-S marker should also be measured, in order that the difference between true north and the measured directions can be taken into account when assessing the orientation of the site. However, the style and direction of the N-S marker varies massively between plans, making automatic identification very difficult. Manual identification of the group of pixels representing the N-S marker has been used in the case studies in sections \ref{sec:CS1} and \ref{sec:CS2}.

\subsubsection{Exclude sparse features}
\label{sec:excl-sparse}

Positively identifying post-hole features directly is difficult without a thorough investigation of the distributions of the shapes of the features identified, which will vary from site to site and depend on such factors as the resolution of the image, the scale of the map, and stylistic choices made by the printers; it is generally far easier to begin with a set of all possible post-holes, and exclude certain features that we are confident are not post-holes. An initial step in this process that is generally extremely effective without tuning the parameter to the particular site is to exclude any `sparse' features from the set of potential post-holes. For present purposes, a sparse feature is one for which, if we were to draw the smallest possible square around the feature, and to count the number of that square's pixels that are coloured black by the feature, the ratio of black to white pixels would be low.

\nb{diagram?}

As an example, consider an `ideal' post-hole object: a solid circle of black cells, with radius $r$, and covering an area of $\pi r^2$ cells. A square bounding this shape would have sides of length $2r$, and cover $4r^2$ cells, so the proportion of the cells in the square that are coloured black is $\pi/4$ (around 0.79). At the opposite end of the scale, the sparsest feature that would require a square of this size to tightly bound it is a straight line of cells, of length $2r$ and width 1, covering an area of $2r$; so the proportion of the cells that are coloured black by the line is $1/2r$, with the covered proportion decreasing as the length of the line increases. We will disregard as noise any cluster of only 1 or 2 cells, so the shortest possible line is 3 pixels long, giving a maximum possible ratio for strictly linear features of 1/3. A sensible starting threshold - and one which works well in practice - cuts midway between these two limits, at $\frac{\pi/4+1/3}{2}$, or about 0.55. 

Between these two extremes there are an infinite variety of possible shapes, each covering a different proportion of its bounding square; the exact threshold to be used can be adjusted to try to capture as much annotation as possible without wrongly classifying too many post-holes. For sites with minimal annotation - such as the Catholme plan investigated in Section \ref{sec:CS2}, which has no text other than the scale marker, and a boundary marked by a single solid line - this single function may be enough to filter out the small, dense post-hole features from the annotations and larger linear features.

\subsubsection{Exclude text and numbers}
\label{sec:closing}

Removing sparse features is unlikely to remove all annotations from the image, particularly if the text is only slightly larger than the post-hole objects themselves, or in a bold font; under these circumstances, the size and density of the characters may be very similar to that of the post-hole features. However, text and numbers have more complex shapes than post-holes, and those that remain after the sparser feature characters (such as the letter l or L, or the number 1) are removed will generally feature closed or nearly-closed loops. We can identify any such features using a morphological closing \cite{Serra1982}, consisting of two translations of the data: given a set of points $X \in \mathbb{R}^2$ and a structuring element $B$,

\begin{description}
\item[Dilation] $\delta_B(X)$ of $X$ by $B$ is the set of points $x \in R^2$ such that the translation of $B$ by $x$ has a non-empty intersection with set $X$.
\vspace{-25pt}
\item[Erosion] $\varepsilon_B(X)$ of $X$ by $B$ is the set of points $x \in R^2$ such that the translation of $B$ by $x$ is included in $X$.
\vspace{-25pt}
\item[Closing] of $X$ by $B$ is given by $\phi_B(X) = \varepsilon_B(\delta_B(X))$ \cite{Vincent1997}.
\end{description}


The data is prepared by converting each of the remaining candidate features into a polygon by dissolving the boundaries between the black pixels, then removing any holes, leaving only an outline of the feature. The set of points $X$ is the set of all points contained within the feature outline, and the structuring element a disc of 1 cell radius: a larger choice of radius would result in more convex features after closing, but we are only interested in smoothing irregularities in the shapes, so a small radius is appropriate here. As shown in Figure~\ref{fig:morph-closing-example}, the dilation $\delta_B(X)$ contains all points that are covered by $B$ when the centre of $B$ is in $X$, while the erosion of $\delta_B(X)$ contains those that are covered by the centre of $B$ when all of $B$ is inside $\delta_B(X)$. Where the initial feature is convex and solid, the boundary is smoothed, but the number of cells covered is not changed, as in Figure~\ref{fig:closing-simple-1}; when the initial feature has concave edges, or a hole, the area covered by the closing will be quite different to the area covered by the original feature. Any features whose closing covers more pixels than its original footprint, as in Figure~\ref{fig:closing-complex-1}, will be excluded from the set of candidate post-holes.

\begin{figure}[h!]
\centering
\caption{Closing of features of various complexity, showing changes to feature boundary after closing with $B$; the hatched area is the new boundary at each step.\\ The footprint of the simpler convex post-hole is unchanged by the procedure; the details of the complex annotation feature are smoothed by the closing, which covers 5 more complete pixels than the original shape.}
\label{fig:morph-closing-example}
%
\begin{subfigure}[b]{0.48\textwidth}
\caption{Simple (convex) feature}
\label{fig:closing-simple-1}
\centering
\includegraphics[scale=0.16]{cl-simple-1-org.pdf}
\includegraphics[scale=0.16]{cl-simple-2-dilated.pdf}
\includegraphics[scale=0.16]{cl-simple-3-eroded.pdf}
\end{subfigure}
%
\begin{subfigure}[b]{0.48\textwidth}
\caption{Complex (concave) feature}
\label{fig:closing-compl-1}
\centering
\includegraphics[scale=0.16]{cl-complex-1-org.pdf}
\includegraphics[scale=0.16]{cl-complex-2-dilated.pdf}
\includegraphics[scale=0.16]{cl-complex-3-eroded.pdf}
\end{subfigure}
\end{figure}





\subsubsection{Fill in broken site boundaries}
\label{sec:site-boundaries}
The boundary of the excavation is generally marked with a solid, dashed or broken (\texttt{-$\cdot$-}) line. Even the shortest line segments will have been identified as sparse features in all cases, removing solid and dashed lines from the set of candidate features, but a broken line can be more problematic. In all but position, the dots are likely to resemble post-holes, but they lie on a straight line; if we accept them as post-holes and measure the angles between them, they will introduce a bias into our data set. However, we can use that very characteristic to distinguish them from post-holes. For all of the sparsest features (say, all features for which less than 20\% of the bounding square is covered, to ensure that only line segments are likely to be included), we extend a line segment through the two most distant points, and identify any adjacent features that lie along this transect, or within an arc $1^\circ$ either side of it. Any feature that lies on two of these transects is assumed to be part of a broken boundary line, and removed from the set of potential post-holes.


%=================================================================================

%\subsubsection{! Assess the shape of the remaining features}

%\nb{Is this even possible? Try boxplots of height, width, sparsity, w/h ratio, abs. size. Give examples.}

%\nb{Also consider: many are conservative (ie. likely to leave points as post-holes, rather than exclude them): won't remove post-holes without good reason. Particularly removal of sparse features, tall features (add function for wide features?), points between annotation marks: all of these should be fairly specific ways of removing only points that aren't post-holes.}

% ====================================================================================

\subsubsection{Additional filtering methods}
\label{sec:alternative-techniques}

While the procedure above is adequate for many sites, there are a number of other techniques that may be useful alongside or in place of those listed above. An approach that is best used when only a few annotations remain among the post-holes is to identify any particularly tall features as annotations. Having obtained the heights of all of the features, we find the upper and lower quartiles, $q_{0.75}$ and $q_{0.25}$; then an `unusually tall' point is defined using a formula often applied to identify outliers in boxplots: an extreme measurement is one that lies above $q_{0.75} + 1.5(q_{0.75} - q_{0.25})$. A less conservative approach, but one that requires a greater degree of tuning, is to apply a simple filter to identify horizontal or vertical strips of black pixels, similar to that applied in \ref{sec:rescale}; the most effective filter size will need to be specified manually depending on the size of font used and the proportions of the site, although good results have been obtained over a number of sites using a vertical filter height of 9 (classifying any features with strips of 9 black pixels as an annotation), or a horizontal filter width of 7.

\subsection{Extract angles between post-holes}
\label{sec:posts-to-angles}

The mean $x$ and $y$ coordinates of the post-holes can easily be obtained from the final feature raster, defining a set of points that represent all of the small, dense features identified in the site. However, not all of those features are necessarily of interest to us. We wish to investigate the orientation of the larger structures whose boundaries are marked by sets of post-holes, not of the post-holes themselves. Some further data cleaning is therefore required to remove those points that do not belong to a group representing any particular structure. In particular, we will remove any points that are particularly isolated, and any points that do not form a line or right-angle with their two nearest neighbours.

\subsubsection{Removal of remote points}
\label{sec:filter-rectilinear}

A fairly conservative approach to distance-based post-hole removal has been used, to ensure that only post-holes that are truly remote from their nearest neighbours are excluded from the data set. To avoid the need to manually specify a site-specific parameter determining an `acceptable' level of remoteness in terms of absolute distance from the nearest point, the approach applied is one commonly used to identify outliers when producing boxplots. Denoting the Euclidean distance from each point $i$ to its nearest neighbour as $\lambda_i$, $i$ is classed as an outlier if $\lambda_i > q_{0.75}(\boldsymbol{\lambda}) + 1.5 \left(q_{0.75}(\boldsymbol{\lambda}) - q_{0.75}(\boldsymbol{\lambda}) \right)$. Such outliers are assumed to be unrelated to any larger structures, so we do not include them in the angular analysis.


\subsubsection{Measuring angles}

Buildings and other structures are generally slightly separated from one another, so post-holes that form part of a wall are generally likely to have as their nearest neighbours other post-holes which are part of the same wall - and so to share a common orientation (modulo $\pi/2$) - while post-holes which do not lie within a structure may have as their nearest neighbour a point lying in any direction. In order to assess the post-holes' degree of alignment, an appropriate subset of the angles between them must be measured.

A potential approach would be to obtain the angles between all points within a certain radius of one another. However, this method depends on the investigator to decide on an appropriate radius within which points are to be included. Where the true scale of the map is known, this may be feasible, although still  reliant on a subjective judgement of `appropriate' or `useful'. It has been suggested \cite{Kendall2014} that walls and other structures may have been based on modules of between 4.5 and 5.5 metres (depending on the geographical location of the site under consideration), so treating points within around 5m of each other as part of the same structure, and measuring the angles between them, seems reasonable. However, for sites where the true scale is not known, estimation of an appropriate radius will be entirely subjective; it would be preferable to use a method that can be universally applied, with no estimating of parameters required.

One such universal method is the one adopted here: a set of angles is obtained by calculating the angle from each point to its single nearest neighbour, using the \textbf{atan2} function defined in (\ref{eqn:atan2}).  This approach has a further advantage in that it will produce a more concentrated distribution of angles than the method discussed in the previous paragraph: even for points that are perfectly aligned along lines at right-angles to one another, taking all angles within a certain radius will necessarily measure angles that do not reflect the dominant axis, as can be seen in Figure \ref{fig:angle-extraction-methods}.

\nb{is figure necessary here?}

\begin{figure}[!h]
\centering
\caption{Comparison of angle extraction methods across a set of points representing the post-holes of a regular structure. Angles are shown on the structure and in a circular plot}
\label{fig:angle-extraction-methods}
%
\begin{subfigure}[t]{0.45\textwidth}
\caption{Angles taken within a certain radius}
\end{subfigure}
%
%
\begin{subfigure}[t]{0.45\textwidth}
\caption{Angles taken between nearest neighbours}
\end{subfigure}
%
%
\end{figure}


\subsubsection{Conversion of axial data into circular data}
Under our null assumption that the measured angles will tend to be concentrated around the four axes of an underlying grid, we consider the raw angles $\phi, \phi + \nicefrac{\pi}{2}, \phi + \pi,$ and $\phi + \nicefrac{3\pi}{2}$ to be part of the same axis, and so we wish to analyse them as the same angle $\theta = \phi \text{ (mod }\nicefrac{\pi}{2})$.
To this end we will follow Fisher's approach to $p$-axial data ~\cite{Fisher1993}, using $p=4$: the raw angles $\phi_i$ are transformed to $\theta_i = 4\phi_i \text{ (mod } 2\pi)$ - equivalently, $\theta_i = 4 \times \left( \phi_i \text{ (mod }\nicefrac{\pi}{2})\right)$. Raw angles $\phi_i$ that share a perpendicular orientation - that is, angles that are directly opposed or perpendicular to one another - are thus mapped to the same angle $\theta_i$,  giving a unimodal data set with support $(0, 2\pi)$, to which we can fit a circular distribution.


 
After a distribution has been fitted to the transformed angles $\mathbf{\theta}$, the mean sample direction obtained will be back-transformed by dividing by 4, to give the direction of one (and hence, trivially, all) of the axes of the grid; to allow for easier interpretation of the back-transformed angles, these will be given in degrees, rather than radians. Measures of dispersion such as the mean resultant length $\bar{R}$ will not be back-transformed, as per Fisher's recommendation, but will be given in terms of the transformed data.

%\begin{figure}[h!]
%\caption{Simulated set of buildings with post-holes 1m apart, with $N(0,0.1)$ perturbation, and associated angles. The \nb{estimated?} kernel density is shown in red. \nb{plot needs to be finalised \& tidied up. Either remove kernel density or mention it!}}
%\centering
%\label{fig:sim1}
%\begin{subfigure}[t]{0.38\textwidth}
%\caption{Simulated set of post-holes}
%\label{fig:sim-plot-1}
%\includegraphics[scale=0.3]{./img/sim-plot-1.pdf}
%\end{subfigure}
%\begin{subfigure}[t]{0.3\textwidth}
%\caption{Raw angles $\boldsymbol{\phi}$}
%\label{fig:sim-q-plot-1}
%\includegraphics[scale=0.3]{./img/sim-q-plot-1.pdf}
%\end{subfigure}
%\begin{subfigure}[t]{0.3\textwidth}
%\caption{Transformed angles $\boldsymbol{\theta}$}
%\label{fig:sim-q4-plot-1}
%\includegraphics[scale=0.3]{./img/sim-q4-plot-1.pdf}
%\end{subfigure}
%\end{figure}





\end{document}