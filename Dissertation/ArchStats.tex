\documentclass[10pt,fleqn]{article} 
% NB. IF SUBMITTING LATEX FILE, WILL NEED TO EITHER PROVIDE STYLE FILE OR PASTE RELEVANT PACKAGES HERE!

% DOUBLE OR SINGLE SPACED? CHECK WHICH WILFRID WOULD PREFER

% NEW PAGE BEFORE EACH CHAPTER?

% CHANGE BIBLIOGRAPHY FORMAT - SHOULDN'T REPORT ARTICLES LIKE THIS
% ALSO: IS THERE A PREFERRED ORDER FOR REFERENCES? ALPHABETICAL/IN ORDER OF APPEARANCE?

%======================================================================
\usepackage{/home/clair/Documents/mystyle}
\usepackage{subfiles}

\addbibresource{../ArchStats-references.bib}
%\AtEveryBibitem{\clearfield{url}}
\AtEveryBibitem{\clearfield{doi}}
\AtEveryBibitem{\clearfield{isbn}}
\AtEveryBibitem{\clearfield{issn}}

%======================================================================

\begin{document}

\section*{Things to code}
\vspace{-15pt}

\todo{Run procedure over Brandon \& compare clustering}

\todo{AIC$_C$ comparison across uniform, VM and JP}

\todo{simulate a larger site: at what point (in terms of $n$) does JP become a more viable alternative than vM? (could give single plot of all buildings, labelled, and sample stats for all of them) - need to take account of the fact that the exact distribution is likely to change with $n$}

\todo{Get wall extraction algorithm - plot robust-regression lines against adjacent points that share the global orientation}

\todo{Run cluster comparison over simulated building data with many buildings}

\section*{Things to mention}
\begin{itemize}

\item 
``Only by a doubling approach can a convenient exponential-family model be arrived at for axial data'' \cite{Arnold2011}

\item
Surely the most important factor in establishing gridding across a whole site is to check that each building shares the same axis? Just showing that all points share an axis is not enough, since this may only show us the dominant axis.
 
\item
Rather than just pairwise tests of `same distribution or not', use GoF tests of global distribution against each subset.

\item
Claim that von Mises has kurtosis = 0: recheck this and find citation with page number.
\end{itemize}

%\tableofcontents

\section*{A note on axial data and notation}
Under our null assumption that the measured angles will tend to be concentrated around the four axes of an underlying grid, we will consider the angles $\phi, \phi + \pi/2, \phi + \pi,$ and $\phi + 3\pi/2$ to be part of the same axis, and so we wish to analyse them as the same angle $\theta = \phi \text{ modulo }\pi/2$.
To this end we will follow Fisher's approach to $p$-axial data \cite{Fisher1993} using $p=4$: the raw angles $\phi_i$ are transformed to $4\theta_i \text{ modulo } 2\pi$ - equivalent to taking $4 \times (\phi_i \text{ modulo }\pi/2)$ giving a unimodal data set to which we can fit a circular distribution. Raw angles that share a perpendicular orientation - that is, angles that are directly opposed or perpendicular to one another - are thus mapped to the same angle, allowing clearer analysis of the direction of the underlying grid.

Once a distribution has been fitted, the mean sample direction obtained will be back-transformed by dividing by 4, to give the direction of one (and hence, trivially, all) of the axes of the grid; to allow for easier comparison between the back-transformed angles, these will be given in degrees, rather than radians. Measures of dispersion such as the mean resultant length $\bar{R}$ will not be back-transformed, as per Fisher's recommendation, but will be given in terms of the transformed data.

\newpage




\subfile{./sections/intro/Introduction.tex}

\subfile{./sections/circular-stats/circ-stats.tex}

\subfile{./sections/gridding/gridding-evidence.tex}

\subfile{./sections/standard-unit/standard-unit-evidence.tex}

\subfile{./sections/data-cleaning/data-cleaning.tex}

\subfile{./sections/CS1-Genlis/Genlis.tex}

\subfile{./sections/CS2-Catholme/Catholme.tex}

\subfile{./sections/simulations/simulations.tex}

\subfile{./sections/conclusions/conclusions.tex}

\subfile{./sections/extensions/extensions.tex}


%======================================================================
\newpage
\printbibliography

\end{document}