\documentclass[10pt,fleqn]{article} 


% NEW PAGE BEFORE EACH CHAPTER?

% CHANGE BIBLIOGRAPHY FORMAT - SHOULDN'T REPORT ARTICLES LIKE THIS
% ALSO: IS THERE A PREFERRED ORDER FOR REFERENCES? ALPHABETICAL/IN ORDER OF APPEARANCE?

%======================================================================
\usepackage{/home/clair/Documents/mystyle}
\usepackage{subfiles}

% style & colour to be used for code listings
\lstset{	basicstyle = \fontsize{10}{13}\selectfont\ttfamily,%\linespread={0.8\listingsfont}\singlespacing
		commentstyle = \color{gray}\textit,
		keywordstyle = \color{black},
		xleftmargin = 0cm,
		lineskip={-2pt},
		showstringspaces=false
	   }
	
\usetikzlibrary{shapes,arrows}

%\doublespacing
%\onehalfspacing

\numberwithin{equation}{section}
\numberwithin{figure}{section}
\numberwithin{table}{section}


\addbibresource{../ArchStats-references.bib}
%\AtEveryBibitem{\clearfield{url}}
\AtEveryBibitem{\clearfield{doi}}
\AtEveryBibitem{\clearfield{isbn}}
\AtEveryBibitem{\clearfield{issn}}

%======================================================================

\begin{document}

\section*{Things to make \& do}
\vspace{-15pt}

\todo{Have I adequately explained WHY each step is necessary?}

\todo{Think a glossary might be a good idea...}

\todo{Add section numbers to code}
\todo{Replace all !h with !ht}
\todo{SHOULD ONLY BE USING PARAMETRIC BOOTSTRAP GOF TESTS!}
\todo{Refer to chapters or sections? Need to decide.}
\todo{Distribution names should be in capitals}
\todo{Check usage of the word \textcolor{red}{feature}. Should be used either to refer to a feature on a raster or a larger structure on the map - not both. Need to clarify terminology.}
\todo{List all required R packages!}

\section*{Things to mention}
\begin{itemize}

\item
Why unimodal, reflectively symmetric models? Mention this in the introduction.

\item
Relatively high bandwidth of 50 is generally used in kernel density estimation: using a relatively high bandwidth of 50 to avoid over-smoothing the density - a real risk in circular data, because of its limited support

\item
Have I mentioned under- or over-fitting anywhere? Think I may have used it incorrectly, if I did. Check this.

\item
Linear plots will generally be used, since these are often easier to interpret; the data will be cut approximately at the antipode, to avoid obscuring the shape at the peak, unless mentioned otherwise.

\item
Further functions (including bootstrap equivalents not required in the case studies) are available in the packages on Github.

\item
Data-cleaning method given is fairly conservative as far as cleaning post-holes is concerned: you can run all steps even when they aren't necessary, and should do only a minimal amount of damage (as in Catholme site) - think I mentioned this in Catholme case study?

\item
von Mises, like the Normal distribution, is a maximum entropy distribution - another reason to make it the natural analogue. For large $\kappa$, tends to a Normal distribution with variance $1/k$ (Wallace2000).

\item
Make clear that a point's orientation is fixed, as an attribute of that point. Any subsetting, clustering etc does not result in re-measurement: $\theta_i$ and $\phi_i$ are fixed for each $i$ but the $i$ are repartitioned.

\item
 - All angular calculations are considered to be modulo $2\pi$ unless stated otherwise.\\
 - $\theta$ are considered to be continuous throughout.
 - A result will be considered significant if it holds at the 5\% level.

\item
Should be wary of reporting things to a very high degree of precision when presenting final results: we're not working from real measurements here, but from a small-scale representation, often one that has been printed \& scanned. Hopefully accurate enough within themselves, but need to bear this in mind when reporting re-scaled distances/angles.

Also gives us a rationale for looking for features along a line: centre-points are too small (single point only), post-holes are larger than this.

\item
Functions don't overwrite the existing object - have to be updated manually.
\textit{(probably mention in appendix with R code for pre-processing functions)
}

\end{itemize}
\newpage

\subfile{./sections/front-matter/Front-matter.tex}

\newpage
\tableofcontents

\todo{Abstract}
\todo{Acknowledgements}

\newpage

\subfile{./sections/intro/Introduction.tex}
\newpage

\subfile{./sections/circular-stats/circ-stats.tex}
\newpage

\subfile{./sections/data-cleaning/data-cleaning.tex}
\newpage

\subfile{./sections/gridding/gridding-evidence.tex}
\newpage

%\subfile{./sections/standard-unit/standard-unit-evidence.tex}
%\newpage

\subfile{./sections/CS1-Genlis/Genlis.tex}
\newpage

\subfile{./sections/CS2-Catholme/Catholme.tex}
\newpage

%\subfile{./sections/simulations/simulations.tex}
%\newpage

\subfile{./sections/conclusions/conclusions.tex}
\newpage

\subfile{./sections/extensions/extensions.tex}
\newpage

%======================================================================
\begin{appendix}
\section{R code}

\subsection{Circular functions}

All circular functions have been adapted from those given in \cite{Pewsey2014}, except where stated otherwise. 

A package containing all of the functions given below can be found at \url{https://github.com/ClairBee/AS.circular}; this (and any of the other packages mentioned) can be downloaded and installed directly from R using \texttt{install\textunderscore github} in the \texttt{devtools} package.

\subsubsection{Jones-Pewsey distribution}
%\lstinputlisting{./code/Circular-functions.R}

\subsubsection{Model fitting and comparison}
%\lstinputlisting{./code/Model-fitting-functions.R}

\subsection{Pre-processing functions}
Packages containing all of the functions listed here can be found at \url{https://github.com/ClairBee/AS.angles} and \url{https://github.com/ClairBee/AS.pre-processing}.
%\lstinputlisting{./code/Pre-processing-functions.R}

\subsection{Case Study 1: Genlis}
\label{app:CS1}
%\lstinputlisting{./code/Genlis.R}

\subsection{Case Study 2: Catholme}
\label{app:CS2}
%\lstinputlisting{./code/Catholme.R}

\end{appendix}

%======================================================================
\newpage
\printbibliography

\end{document}