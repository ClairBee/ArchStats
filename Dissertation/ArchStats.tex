\documentclass[10pt,fleqn]{article} 
% NB. IF SUBMITTING LATEX FILE, WILL NEED TO EITHER PROVIDE STYLE FILE OR PASTE RELEVANT PACKAGES HERE!

% DOUBLE OR ONE-AND-A-HALF SPACED? CHECK WHICH WILFRID WOULD PREFER

% NEW PAGE BEFORE EACH CHAPTER?

% CHANGE BIBLIOGRAPHY FORMAT - SHOULDN'T REPORT ARTICLES LIKE THIS
% ALSO: IS THERE A PREFERRED ORDER FOR REFERENCES? ALPHABETICAL/IN ORDER OF APPEARANCE?

%======================================================================
\usepackage{/home/clair/Documents/mystyle}
\usepackage{subfiles}

\doublespacing
%\onehalfspacing

\addbibresource{../ArchStats-references.bib}
%\AtEveryBibitem{\clearfield{url}}
\AtEveryBibitem{\clearfield{doi}}
\AtEveryBibitem{\clearfield{isbn}}
\AtEveryBibitem{\clearfield{issn}}

%======================================================================

\begin{document}

\section*{Things to code}
\vspace{-15pt}

\todo{Run procedure over Brandon \& compare clustering}

\todo{simulate a larger site: at what point (in terms of $n$) does JP become a more viable alternative than vM? (could give single plot of all buildings, labelled, and sample stats for all of them) - need to take account of the fact that the exact distribution is likely to change with $n$}

\todo{Get wall extraction algorithm - plot robust-regression lines against adjacent points that share the global orientation \nb{surely robust regression isn't even necessary, since I know the expected angle? Just plot a line at that point instead}}

\todo{Run cluster comparison over simulated building data with many buildings}

\todo{Refer to chapters or sections? Need to decide.}

\todo{Check `in search of quanta' - how was perturbation done? What did Giacomo do? And how concrete/quantified were the conclusions there?}

\todo{Split out procedure(start to finish) as a separate section?}

\todo{Think a glossary might be a very good idea...}


\section*{Things to mention}
\begin{itemize}

\item 
``Only by a doubling approach can a convenient exponential-family model be arrived at for axial data'' \cite{Arnold2011}

\item
Required level of significance: generally 5\% (also for confidence intervals)

\item
All angular calculations are considered to be modulo $2\pi$ unless stated otherwise.

\item
$\theta$ are considered to be continuous throughout.

\item
Should be wary of reporting things to a very high degree of precision when presenting final results: we're not working from real measurements here, but from a small-scale representation, often one that has been printed \& scanned. Hopefully accurate enough within themselves, but need to bear this in mind when reporting re-scaled distances/angles.

Also gives us a rationale for looking for features along a line: centre-points are too small (single point only), post-holes are larger than this.

\item
scale-based parameters are not generally useful, since scale is very inaccurately estimated in the plan (generally \nb{n} pixels represent 1m). Also some maps may not be rescaled, eg. if the plan doesn't have a scale marker..

\end{itemize}
\newpage

\subfile{./sections/front-matter/Front-matter.tex}

\newpage
\tableofcontents

\todo{Abstract}
\todo{Acknowledgements}

\newpage

\subfile{./sections/intro/Introduction.tex}
\newpage

\subfile{./sections/circular-stats/circ-stats.tex}
\newpage

\subfile{./sections/data-cleaning/data-cleaning.tex}
\newpage

\subfile{./sections/gridding/gridding-evidence.tex}
\newpage

\subfile{./sections/standard-unit/standard-unit-evidence.tex}
\newpage

\subfile{./sections/CS1-Genlis/Genlis.tex}
\newpage

\subfile{./sections/CS2-Catholme/Catholme.tex}
\newpage

\subfile{./sections/simulations/simulations.tex}
\newpage

\subfile{./sections/conclusions/conclusions.tex}
\newpage

\subfile{./sections/extensions/extensions.tex}


%======================================================================
\newpage
\printbibliography

\end{document}