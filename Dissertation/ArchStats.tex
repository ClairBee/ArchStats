\documentclass[10pt,fleqn]{article} 


% NEW PAGE BEFORE EACH CHAPTER?

% CHANGE BIBLIOGRAPHY FORMAT - SHOULDN'T REPORT ARTICLES LIKE THIS
% ALSO: IS THERE A PREFERRED ORDER FOR REFERENCES? ALPHABETICAL/IN ORDER OF APPEARANCE?

%======================================================================
\usepackage{/home/clair/Documents/mystyle}
\usepackage{subfiles}

\graphicspath{{../sections/CS1-Genlis/img}
			  {../sections/CS2-Catholme/img}
			  {../sections/Gridding/img}
			  {../sections/Circular-stats/img}} 

% style & colour to be used for code listings
\lstset{	basicstyle = \ttfamily\footnotesize,
		commentstyle = \color{gray}\textit,
		keywordstyle = \color{black},
		xleftmargin = 0cm
	   }
	
%\doublespacing
%\onehalfspacing

\numberwithin{equation}{section}

\addbibresource{../ArchStats-references.bib}
%\AtEveryBibitem{\clearfield{url}}
\AtEveryBibitem{\clearfield{doi}}
\AtEveryBibitem{\clearfield{isbn}}
\AtEveryBibitem{\clearfield{issn}}

%======================================================================

\begin{document}

\section*{Things to make \& do}
\vspace{-15pt}

\todo{Change PP functions to allow them to be added to existing plot - can overlay von Mises, Jones-Pewsey, and mixture models. Also should extract residuals.}

\todo{EM classifier is a "winner takes all" approach}

\todo{Calculate mean squared error \& use to obtain residuals - should be able to give graphical plot of these to assess fit/support numeric assessment. If I would use them in data on R, I should use them with data for the unit circle.}

\todo{Change hyperref settings so that links are highlighted as a different colour, not just by drawing a box around them}

\todo{\nb{Regional clustering!} Procedure: divide site into grid according to whatever distance. (Will need to compare results with various distances to show that same regions are highlighted regardless). Get mean direction in each grid square. Cluster. Test clusters to ensure that common mean is plausible. Treat all angles in cluster as subsample \& get distribution (can't test for common concentration etc because many squares may contain only one or two points). If all else fails, pick all means that are within a certain range of global mean (possibly as an absolute range either side, rather than a \% - but need to test to decide best approach)}

\todo{Add circular standard deviation to section on population measures, if using for clustering. Make the point that each of these measures will be used for different purposes, but our main descriptive measure will still be $\bar{R}$. May also need to put in a measure of circular distance? Although I think I've used two different measures for different things, so that may not be necessary - if that's the case, can just define them where they're used.}

\todo{Change sub-caption font size: should be smaller}

\todo{Code font \& size}

\todo{$i$ denotes an observation: a point or angle. Groups should be denoted $j$. Or vice versa. Doesn't matter, just make sure this is consistent throughout.}

\todo{Have I adequately explained WHY each step is necessary?}

\todo{Sort out spacing \& inline formatting on equations}

\todo{Distribution names: capitals or not?}

\todo{Run procedure over Brandon \& compare clustering}

\todo{Can I shrink subscript fonts?}

\todo{Need to be careful about ML vs BC estimators. Either use tilde vs hat or be stricter with BC subscript. Need to differentiate somehow.}

\todo{Create a mixture uniform \& vM with high $\kappa$: what is this identified as? Which model fits it better \& at which points in the distribution \nb{(use a PP/QQ plot to explore this)}}

\todo{simulate a larger site: at what point (in terms of $n$) does JP become a more viable alternative than vM? (could give single plot of all buildings, labelled, and sample stats for all of them) - need to take account of the fact that the exact distribution is likely to change with $n$}

\todo{Get wall extraction algorithm - plot robust-regression lines against adjacent points that share the global orientation \nb{surely robust regression isn't even necessary, since I know the expected angle? Just plot a line at that point instead}}

\todo{Run cluster comparison over simulated building data with many buildings}

\todo{Refer to chapters or sections? Need to decide.}

\todo{Think a glossary might be a very good idea...}

\todo{Check any functions using atan2 to make sure everything is passed as y, x. Otherwise, angles are measured the wrong way round.}

\section*{Things to mention}
\begin{itemize}

\item
von Mises, like the Normal distribution, is a maximum entropy distribution - another reason to make it the natural analogue. For large $\kappa$, tends to a Normal distribution with variance $1/k$ (Wallace2000).

\item
Make clear that a point's orientation is fixed, as an attribute of that point. Any subsetting, clustering etc does not result in re-measurement: $\theta_i$ and $\phi_i$ are fixed for each $i$ but the $i$ are repartitioned.

\item
List all required R packages!

\item
A summary `cheat sheet' showing the outline of the procedure, with appropriate R functions, is given in Appendix ??.

\item
Sizes of post-holes on maps are not necessarily representative of the sizes of the genuine post-holes \nb{at least, I assume they can't be - waiting on Chris S to confirm}, so can't use our knowledge of their expected proportions to determine an appropriate set of parameters: depends on scaling used by printer. Not to mention the resolution of the JPEG.

\item 
``Only by a doubling approach can a convenient exponential-family model be arrived at for axial data'' \cite{Arnold2011}

\item
Required level of significance: generally 5\% (also for confidence intervals)

\item
All angular calculations are considered to be modulo $2\pi$ unless stated otherwise.

\item
$\theta$ are considered to be continuous throughout.

\item
Should be wary of reporting things to a very high degree of precision when presenting final results: we're not working from real measurements here, but from a small-scale representation, often one that has been printed \& scanned. Hopefully accurate enough within themselves, but need to bear this in mind when reporting re-scaled distances/angles.

Also gives us a rationale for looking for features along a line: centre-points are too small (single point only), post-holes are larger than this.

\item
Functions don't overwrite the existing object - have to be updated manually.
\textit{(probably mention in appendix with R code for pre-processing functions)
}

\item
scale-based parameters are not generally useful, since scale is very inaccurately estimated in the plan (generally \nb{n} pixels represent 1m). Also some maps may not be rescaled, eg. if the plan doesn't have a scale marker..

\end{itemize}
\newpage

\subfile{./sections/front-matter/Front-matter.tex}

\newpage
\tableofcontents

\todo{Abstract}
\todo{Acknowledgements}

\newpage

\subfile{./sections/intro/Introduction.tex}
\newpage

\subfile{./sections/circular-stats/circ-stats.tex}
\newpage

\subfile{./sections/data-cleaning/data-cleaning.tex}
\newpage

\subfile{./sections/gridding/gridding-evidence.tex}
\newpage

%\subfile{./sections/standard-unit/standard-unit-evidence.tex}
%\newpage

\subfile{./sections/CS1-Genlis/Genlis.tex}
\newpage

\subfile{./sections/CS2-Catholme/Catholme.tex}
\newpage

\subfile{./sections/simulations/simulations.tex}
\newpage

\subfile{./sections/conclusions/conclusions.tex}
\newpage

\subfile{./sections/extensions/extensions.tex}
\newpage

%======================================================================
\begin{appendix}
\section{R code}

\subsection{Circular functions}

All circular functions have been adapted from those given in \cite{Pewsey2014}, except where stated otherwise. 

A package containing all of the functions given below can be found at \url{https://github.com/ClairBee/AS.circular}; this (and any of the other packages mentioned) can be downloaded and installed directly from R using \texttt{install\textunderscore github} in the \texttt{devtools} package.

\subsubsection{Jones-Pewsey distribution}
\lstinputlisting{./code/Circular-functions.R}

\subsubsection{Model fitting}
\lstinputlisting{./code/Model-fitting-functions.R}

\subsubsection{Model comparison}
\lstinputlisting{./code/Model-fitting-functions.R}

\subsection{Pre-processing functions}
Packages containing all of the functions listed here can be found at \url{https://github.com/ClairBee/AS.angles} and \url{https://github.com/ClairBee/AS.pre-processing}.
\lstinputlisting{./code/Pre-processing-functions.R}

\subsection{Case Study 1: Genlis}
\label{app:CS1}
\lstinputlisting{./code/Genlis.R}

\subsection{Case Study 2: Catholme}
\label{app:CS2}
\lstinputlisting{./code/Catholme.R}

\subsection{Simulations}
\lstinputlisting{./code/Simulations.R}

\end{appendix}

%======================================================================
\newpage
\printbibliography

\end{document}